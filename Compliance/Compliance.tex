\documentclass[a4paper,10pt]{scrartcl}
\usepackage[utf8]{inputenc}
\usepackage{graphicx}
\usepackage{subfig}
\usepackage{amsmath}
\usepackage{geometry}
\geometry{a4paper,left=40mm,right=30mm, top=1cm, bottom=2cm} 
%opening
\title{TGI-1}
\author{}

\begin{document}
\section{Einführung in Grundbebriffe}

\textbf{IT-Compliance: }  IT-Compliance bezeichnet die Kenntnis und
Einhaltung sämtlicher regulatorischer
Vorgaben und Anforderungen an das
Unternehmen, die Aufgabe und Einrichtung
entsprechender Prozesse und die Schaffung
eines Bewusstseins der Mitarbeiter für
Regelkonformität, sowie die Kontrolle und
Dokumentation der Einhaltung der
relevanten Bestimmungen gegenüber
internen und externen Adressaten.
\\
\\
\textbf{IT-Governence: } Liegt der Verantwortung des Vorstands und des Managements und ist wesentlicher Bestandsteil der 
Unternehmungsführung. IT-Governence besteht aus Führung, Organisiationsstrukturen und Prozessen, die sicherstellen,
das die IT die Unternehmenstrategie und Ziele unterstüzt.
\\
\textbf{ISMS = Informationssicherheitsmanagementssystem: } Beschreibt das allgemeine Sicherheitsmanagement speziell im Bereich der
Informationssicherheit. ISMS es ein komplexer Prozeß der Steuerung von materiellen, konzeptionellen und menschlichen Ressourcen mit
dem Ziel, den Anforderungen an die Aspekte -Auftragserfüllung, Vertraulichkeit, Integrität und Verfügbarket einer Organisation
angemessen zu entsprechen.

\textbf{Bedrohung:}
Ereignisse oder Begebenheiten aus dennen ein Schaden entstehen kann.\\

\textbf{Bedrohungskategorie}\\
höhere Gewalt\\
elementare Bedrohung\\
techniches Versagen\\
vorsätzliches Handln\\
menschliche Fehlentscheidung

\textbf {Schwachstelle:} Sicherheitsrelvanter Fehler eines IT-Systems oder eines Prozesses.

\textbf{Schutzmaßnahmen: } Maßnahmen um einen Zustand von Sicherheit zu erreichen oder zu verbessern.
\\
\textbf{Begriffe im Zusammenhang:} Eine Bedrohung nutzt Schwachstellen aus um Assests anzugreifen.
nach BSI:\\
Infrastruktur: Verschlossene Türen + Videokameras\\
Hardware und Software: Firewall, Malewareschutz, IDS (Analyisert und schlägt Alarm wenn Angriff stattfindet) - IPS (leitet sogar noch
GegenMaßnahmen ein)\\
Organisation: Verantwortlichkeiten regeln, Nutzungsverbot nicht freigebender Hardware/Software\\
Kommunikation: Dokumentation der Verkabelung, Regelmäßiger Sicherheitscheck der Netze, restriktive Rechtvergabe\\
Notfallvorsorge: Regelmäßige Datensicherung, TKA-Basisanschluss für Notrufe, Übersichert über Verfügbarkeitsanforderungen\\
Personal: Vertretungsregelung, Awarenessmaßnahmen, Einarbeitung von Mitarbeitern\\

\textbf{Schutzziele: } Generische Sicherheitsziele zur Auswahl von
Maßnahmen und Gestaltung eines
Sicherheitskonzeptes.

\section{Einordnung der Managmentsysteme}

\subsection{Eine Beschreibung nach COBIT 5}

\textbf{It Governence: } Überwacht die IT bezüglich Strategie und Ziele des Unternehmen

\textbf{ISMS = Informationssicherheitmanagementsystem: } Sorgt dafür das Sicherheit der Schutziele garantiert ist - steuert die IT um Informationssicherheit 
zu gewährleisten.

\textbf{IT-Risikomanangent: } Der Teil des ISMS der sich mit der IT beschäftigt (Berichtet an Riskomanagement)

\textbf{IT - Compliance: } Wird von IT-Risikomanagement überwacht und dient zur Umsetzung aller wichtigen und relavanten Maßnahmen die durch
interne so wie externe Anforderungen enstehen (Ist Teil von Compliance)

\section{ISO-Standard}


\textbf{27000} Enthält Überblick und verwendete Definitionen\\
Enthält einen Überblick über die Familie des ISMS-Standards. Sowie
eine Einfühung was ein ISMS (Information Management System ) überhaupt ist.
Eine kurze Beschreibunb von Plan-Do-Check-Act (PDCA). Sowie die Therme und Definitionen
die benötigt werden.
\\
\textbf{27001} Vorraussetzungen\\
Enthält Informationen über den Aufbau, Betrieb, Verbesserungen, Einschätzungen und Behandlung von Risiken eines ISMS.\\
\textbf{27002} Verfahrensregeln für die Informationssicherheit\\
Enthält Sicherheitsmaßnahmen die in 13 unterschiedliche Domänen geteilt werden - Hinweise zum Organisationsaufbau
wie man Richtlinien und Policyerstellung und technische Maßnahmen.\\
\textbf{27003} Umsetzungs-Leitfaden\\
Wie man Projekte aufsetzt - zuerst Management-Zustimmung, ISMS Scope und Richtlinien, Analyse der Organisation, Risikomanagementprozess, Design des ISMS
\textbf{27004} Überprüfen der Wirksamkeit\\
Wirksamkeit des ISMS und der Sicherheitsmaßnahmen\\
\textbf{27005} Informationssicherheit Risikomanagement\\
1. Den Kontext feststellen\\
2. Risiko Einschätzung\\
3. Risiko Behandlung\\
4. Risiko Akzeptanz\\
5. Risiko Kommunikation\\
6. Risiko monitoring und Rezension\\
\section{27001}
7 Schritte:\\
1. Kontext der Organisation\\
2. Leitung\\
3. Planung\\
4. Support\\
5. Betrieb\\
6. Leistungsauswertung\\
7. Verbesserung\\
1.\\
Für den Kontext der Organisation was sind die Hauptziele der Organisation und wie kann
die IT dabei helfen? - Was ist absolut nötig damit der Laden läuft?\\
Welche Anforderungen stellt das Unternehmen deshalb an die IT.\\
Welche Anforderungsgruppen gibt es und welche Anforderungen haben sie (IT-Compliance).\\
2.\\
Management soll Leitung und Bekenntnis zum ISMS zeigen. Viel Bla bla bla
was Management machen soll - Policy - Integrierung des ISMS in Geschäftsprozesse - Förderung und Verbesserung des ISMS - Unterstützen der anderen re... Management\\
3.\\
Unter berücksichtigung der Kontextfeststellung, insbesondere der Anforderungsfeststellung:\\
-Information Security Risk Assement nach Iso 27005: Risiko-Identifizierung, Risiko-Abschätzung, Risiko-Bewertung\\
-Information Security Risk Treatment nach Iso 27005
4.\\
Ressourcen bereitstellen\\
Identifizierung der nötigen Kompetenz\\
Dokumentation zur Feststellung der Kompetenz (Test wäre viel besser)\\
Angstellte sollten Informationssicherheitspolicy kennen (Awareness) und sich ihres
Beitrags zur Sicherheit bewusst sein\\
Komunikation muss klar geregelt werden\\
Dokumentation\\
5.\\
Planen + umsetzen und dokumentieren der Prozesse\\
Umsetzen der Maßnahmen die identifiziert worden\\
Änderungsmanagement bei jedem change\\
ausgelagerte Prozesse müssen auch kontrolliert werden\\

Planung Support und Betrieb im Kreislaufen lassen plötzliche Änderung des Bildes -.-\\
6.\\
Ziel: Bewerten der Effektivität des ISMS Was gemessen Wie Wann Wer - Wann auswertung - Wer soll analysieren?\\
In Betracht ziehen für ob geeignet effektive und angemessenheit: Status vergangener Beschlüsse\\
Ergebnisse interner Audits\\
Feedback von Beteiligten\\
Ergebnisse Risikoanalyse und deren Status\\
Gelegenheiten für Verbesserungen\\
7.\\
Bei nicht Konformität Maßnahmen kontrollieren
und korregieren und Konesequenzen behandln\\
Ursachen herausfinden und beseitigen\\
Gab es da schonmal was ähnliches?\\

\end{document}