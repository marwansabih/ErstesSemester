\documentclass[a4paper,10pt]{scrartcl}
\usepackage[utf8]{inputenc}
\usepackage{graphicx}
\usepackage{subfig}
\usepackage{amsmath}
\usepackage{geometry}
\geometry{a4paper,left=40mm,right=30mm, top=1cm, bottom=2cm} 
%opening
\title{TGI-1}
\author{}

\begin{document}
\section{Einführung in Grundbebriffe}

\textbf{IT-Compliance: }  IT-Compliance bezeichnet die Kenntnis und
Einhaltung sämtlicher regulatorischer
Vorgaben und Anforderungen an das
Unternehmen, die Aufgabe und Einrichtung
entsprechender Prozesse und die Schaffung
eines Bewusstseins der Mitarbeiter für
Regelkonformität, sowie die Kontrolle und
Dokumentation der Einhaltung der
relevanten Bestimmungen gegenüber
internen und externen Adressaten.
\\
\\
\textbf{IT-Governence: } Liegt der Verantwortung des Vorstands und des Managements und ist wesentlicher Bestandsteil der 
Unternehmungsführung. IT-Governence besteht aus Führung, Organisiationsstrukturen und Prozessen, die sicherstellen,
das die IT die Unternehmenstrategie und Ziele unterstüzt.
\\
\textbf{ISMS = Informationssicherheitsmanagementssystem: } Beschreibt das allgemeine Sicherheitsmanagement speziell im Bereich der
Informationssicherheit. ISMS es ein komplexer Prozeß der Steuerung von materiellen, konzeptionellen und menschlichen Ressourcen mit
dem Ziel, den Anforderungen an die Aspekte -Auftragserfüllung, Vertraulichkeit, Integrität und Verfügbarket einer Organisation
angemessen zu entsprechen.

\textbf{Bedrohung:}
Ereignisse oder Begebenheiten aus dennen ein Schaden entstehen kann.\\

Bedrohungskategorie\\
höhere Gewalt\\
elementare Bedrohung\\
techniches Versagen\\
vorsätzliches Handln\\
menschliche Fehlentscheidung

\textbf {Schwachstelle:} Sicherheitsrelvanter Fehler eines IT-Systems oder eines Prozesses.

\textbf{Schutzmaßnahmen: } Maßnahmen um einen Zustand von Sicherheit zu erreichen oder zu verbessern.
\\
\textbf{Begriffe im Zusammenhang:} Eine Bedrohung nutzt Schwachstellen aus um Assests anzugreifen.
nach BSI:\\
Infrastruktur: Verschlossene Türen + Videokameras\\
Hardware und Software: Firewall, Malewareschutz, IDS (Analyisert und schlägt Alarm wenn Angriff stattfindet) - IPS (leitet sogar noch
GegenMaßnahmen ein)\\
Organisation: Verantwortlichkeiten regeln, Nutzungsverbot nicht freigebender Hardware/Software\\
Kommunikation: Dokumentation der Verkabelung, Regelmäßiger Sicherheitscheck der Netze, restriktive Rechtvergabe\\
Notfallvorsorge: Regelmäßige Datensicherung, TKA-Basisanschluss für Notrufe, Übersichert über Verfügbarkeitsanforderungen\\
Personal: Vertretungsregelung, Awarenessmaßnahmen, Einarbeitung von Mitarbeitern\\

\textbf{Schutzziele: } Generische Sicherheitsziele zur Auswahl von
Maßnahmen und Gestaltung eines
Sicherheitskonzeptes.

\section{Einordnung der Managmentsysteme}

\subsection{Eine Beschreibung nach COBIT 5}

\textbf{It Governence: } Überwacht die IT bezüglich Strategie und Ziele des Unternehmen

\textbf{ISMS = Informationssicherheitmanagementsystem: } Sorgt dafür das Sicherheit der Schutziele garantiert ist - steuert die IT um Informationssicherheit 
zu gewährleisten.

\textbf{IT-Risikomanangent: } Der Teil des ISMS der sich mit der IT beschäftigt (Berichtet an Riskomanagement)

\textbf{IT - Compliance: } Wird von IT-Risikomanagement überwacht und dient zur Umsetzung aller wichtigen und relavanten Maßnahmen die durch
interne so wie externe Anforderungen enstehen (Ist Teil von Compliance)





\end{document}