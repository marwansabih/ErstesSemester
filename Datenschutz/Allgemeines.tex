\documentclass[a4paper,10pt]{scrartcl}
\usepackage[utf8]{inputenc}
\usepackage{graphicx}
\usepackage{subfig}
\usepackage{amsmath}
\usepackage{geometry}
\geometry{a4paper,left=40mm,right=30mm, top=1cm, bottom=2cm} 
%opening
\title{TGI-1}
\author{}

\begin{document}
\section{Aktuelle Bedeutung}

\subsection{Wesentliche Datensammler}

\begin{itemize}
 \item Der Staat (Videoüberwachung - Überwachung von Handys, Mailboxen, Emails, Websiten, DNA-Analyse, Biometrische Verfahren, Gesundheitskarte
 \item Wirtschaft: Werbung - erstellen von Kundenprofilen, Austausch von Beschäftigungsdaten, Adresshandel
\end{itemize} 

\subsection{Womit werden Daten gesammelt?}

\begin{itemize}
 \item Internet (kostenlose Emaildienste, Preisauschreiben, Kundenkarten, Cookies)
 \item Google (Google Anaylitcs, Streetview, Booksearch)
 \item Kundenkarten Supermarkt - modernes Rabattsystem
\end{itemize}

\subsection{Warum werden Daten gesammelt?}

\begin{itemize}
 \item Für Werbung: Zielgenauigkeit - damit gibt es weniger Streuverluste
\end{itemize}

\subsection{Wer sind die Datensammler?}

\begin{itemize}
 \item Adresshandel: Fa. Schober Information Group - Bestand in Deutschland 
 50 Millionen Namen und 27 Millionen Haushalte
\end{itemize}


\subsection{Warum braucht der Staat die Daten?}

\begin{itemize}
 \item Pro: Für mehr Sicherheit / Prävention von Kriminalität
 \item Contra: verstärkte Kontrolle/ Verdächtigungen möglich/ Aufhebung der Indivitualität/
 Missbrauch
\end{itemize}

\subsection{Volkzählungsgesetz 25.03.1982}

\begin{itemize}
 \item wollten sehr sehr viel wissen unter anderem Erwerbsleben ... Lebensunterhalt
 etc. etc
 \item Verwendung der Daten für wissenschaftliche Zwecke, Ablgeich Medleregister,
 Regionalplanung, Übermittelung staatliche Ämter
\end{itemize}

\subsection{Urteil BVerfG Volkszählung}
\begin{itemize}
 \item Leitsatz 1: Bedingung der modernen Datenverarbeitung - Schutz des 
 Einzelnen gegen unbegrenzte Weitergabe seiner persönlichen Daten -
 abgeleitet vom allgemeinen Persönlichkeitsrecht des 2 Art. 2 Abs 1 GG
 mit Aritkel 1 Abs. 1 GG
 
 Das Grundrecht auf Informationelle Selbstbestimmung - gewährleistet 
 Befugnis des Einzelnen, selbst über die Preisgabe und Verwendung
 seiner persönlichen Daten zu bestimmen.
\end{itemize}
\subsection{wichtiges aus dem ersten Leitsatz}
\begin{itemize}
 \item Moderne Datenverarbeitung
 \item Schutz des Einzelnen, vor ... persönlicher Daten
 \item allgemeines Persönlichkeitsrecht (Art 2 Abs. 1 GG)
 \item (Art 1 Abs 1 GG) Menschenwürde
 \item Befugnis des Einzelnen, über die Preisgabe und Verwendung
 seiner persönlichen Daten selbst zu bestimmen.
\end{itemize}
\subsection{Urteil BVerfG Volkszählung}
\begin{itemize}
 \item Artikel 2 GG: Freie Entfaltung der Persönlichkeit:\\
 Jeder hat das Recht auf die freie Entfaltung seiner Persönlichkeit,
 soweit er nicht die Rechte anderer verletzt und nicht gegen die
 verfassungsmäßige Ordnung oder das Sittengesetz verstößt (...)
 \item Artikel 1 des GG Schutz der Menschenwürde: Die Würde des Menschen 
 ist unantastbar, Sie zu achten und zu schützen ist Verpflichtung aller
 staatlichen Gewalt (...)
\end{itemize}
\subsection{Urteil BVerfG Volkszählung (15.12.1983)}
\begin{itemize}
 \item Leitsatz 2: Einschränkung nur im überwiegenden Allgemeinintresse
 zulässig
 \item Bedürfen gesetzlicher Grundlage, die der Normenklarheit entspricht
 \item Grundsatz der Verhältnismäßigkeit ist zu beachten
 \item Leitsatz 3: Unterscheidung in personenbezogenen Daten und solche,
 die für statistische Zwecke bestimmt sind.
 \item Stichworte: Personenbezogene Daten sind individualisiert - nicht anonym erhoben und verarbeitet
 \item statistische Daten: keine enge und konkrete Zweckbindung der Daten erforderlich
 \item Schranken der Informationserhebung und Verarbeitung
\end{itemize}

\subsection{Urteil BVerfG Volkszählung (15.12.1983)}

\begin{itemize}
 \item Leitsatz 4 (in Kurz): Volkszählung entspricht den Geboten der Normenklarheit und
 der Verhältnismäßigkeit. - Aber es bedarf zur Sicherung des Rechts auf informationelle
 Selbstbestimmung ergänzende verfahrensrechtliche Vorkehrungen - für Durchführung und 
 Organisation der Datenerhebung.
\end{itemize}

\subsection{Urteil BVerfG Volkszählung (15.12.1983)}
\begin{itemize}
 \item Melderegisterabgleich verstößt gegen das allgemeine Persönlichkeitsrecht (man kann nicht
 alle Daten nicht einfach an alle Ämter senden) Wissenschaftliche Verwendung war ok.
\end{itemize}

\subsection{Urteil VerfG Volkszählung}
\begin{itemize}
 \item Art 2 GG: Allgemeine Persönlichkeitsrecht
 \item Recht auf informationelle Selbstbestimmung - Befugnis des einzelnen, grundsätzlich
 selbst über die Preisgabe und Verwendung seinder persönlichen Daten zu bestimmen.
 \item Schranken für die Erhebung personenbezogener Daten
\end{itemize}

\subsection{Urteil VerfG Volkszählung}
\begin{itemize}
 \item Gesetz besteht aus 
 \item 1. Verwendungszweck der Erhebung: Erhebung muss für den Zweck erforderlich sein (Verhältnismäßigkeit)
 \item 2. Verknüpfungs  und Verwendungsmöglichkeiten: Müssen auf gesetzlich bestimmten Zweck beschränk sein (Normenklarheit)
 \item 3. Aufklärungs und Auskunftspflicht
\end{itemize}

\subsection{Struktur des BDSG}
\begin{itemize}
 \item Allgemeine Vorschriften §§ 1 - 11 BDSG
 \item §§ 12 - 26 Datenverarbeitung der öffentlichen Stellen
 (12-18 Rechtsgrundlage der Datenverarbeitung) (19-21 Rechte der Betroffenen) (22-26 Bundesdatenschutzbeauftragter) 
 \item §§ 27 - 38a BDSG Datenverarbeitung nicht öffentlicher Stellen
\end{itemize}

\subsection{Wann findet das BDSG Anwendung?}
\begin{itemize}
 \item Zweck und Anwendungbereich des Gesetzes: Zweck dieses Gesetzes ist es, den einzelnen
 davor zu schützen, dass er durch den Umgang mit seinen personenbezogene Daten in seinem
 Persönlichkeitsrecht beinträchtigt wird.
 \item Was sind personenbezogene Daten ? Einzelangaben über - persönliche oder sachliche Verhältnisse - die auf den
 Betroffenen bezogen werden können
 \item Gegenteil anonymisierte Daten - die keinen Bezug zu einer Person aufweisen
 \item personenbezogene Daten: Persönliche Grunddaten (Name, Anschrift, Geburtsdatum, Alter, Nationalität)
 \item Daten über Wohnverhältnisse (Inhaber einer Sozial- oder Werkwohnung)
 \item Daten über Einkommen und Vermögen
 \item Identifikationsdaten, Codenummer
\end{itemize}

\subsection{Was ist eine IP-Adresse?}
\begin{itemize}
\item Jeder Rechner im Netz muss über eine IP-Adresse (Internet-Protocol) 
verfügen, jede Ressource im Web wird durch eine URL (Uniform Ressource 
Locator) eindeutig bezeichnet.

Beide Daten werden bei einer Kommunikation als Kopfzeile vorangestellt, sie 
werden an jedem Netzknoten, den ein Datenpaket durchläuft, gelesen und 
temporär festgehalten

\item Ob IP-Adresse personenbezogen ist wird kontrovers disktutiert kann nur differenziert beantwortet werden
je nach Beteiligten.
\item bei Access-Providern ist die IP-Adresse einzelnen Nutzern zuzuordnen->
Access-Provider verfügen überdie Bestandsdaten ihrer Kunden und können so auch
temporär vorgegene IP der Person zuorndnen.
\item Inhaltsanbieter (Hosting Services - könnnen nur mit zusätzlichen Daten
dynamische IP zuordnen)
\item trotzdem sind Dynamische IP-Adressen als personbeziehbare Daten den Regelungen
des Datenschutzrechts zu unterwerfen.
\item Nur wenn definitv ausgeschlossen werden kann, das Rückschlüsse auf die Person gezogen
werden kann - sind dynamische IP's nicht Personenbezogen- d.h es dürfen keine logDatein
etc. bestehen.
\end{itemize}

\section{Besondere Arten personenbezogener Daten (besonders schützenswert).}
\begin{itemize}
 \item rassische, ethnische Herkunft
 \item poltische Meinung
 \item religiöse oder philosophische Überzeugung
 \item Gewerkschaftszugehörigkeit
 \item Gesundheit 
 \item Sexualleben
 
 \item AGG klärt wenn Forderungen durch krankheitsbedingte Fehlzeiten entstehen.
\end{itemize}

\section{Was ist an denen so besonders?}


Dieser Datentyp verlangt erschwerte Form der Verarbeitung - Ohne Einwilligung des
Betroffnen dürfen Sie bei der Datenverarbeitung für eigene Geschäftszwecke nur
für lebenswichtige Interessen oder zum Geltendmachung rechtlicher Ansprüche erhoben
werden.

\section{Wann findet BDSG Anwendung?}

\begin{itemize}
 \item Erheben: Beschaffen von Daten
 \item Verarbeiten: Speichern, Verändern, Übermitteln, Sperren, Löschen
 \item Nutzen: Verwendung die nicht Verarbeitung ist.
\end{itemize}

\section{Öffentliche Stellen des Bundes?}

\begin{itemize}
 \item Behörden (Ober-, Mittel- und Unterbehörden), 
 \item Organe der Rechtspflege (alle Gerichte)
 \item andere öffentlich-rechtlich organisierte Einrichtungen des Bundes
 \item bundesunmittelbare Körperschaften
 \item Anstalten (Bundesbank) und Stiftungen (Stiftungen des öffentlichen Rechts)
 \end{itemize}
 \section{Was sind öffentliche Stellen der Länder?}
 -Ausführung von Bundesrecht\\
 -Organe der Rechtspflege\\
 
 Kommt meist nicht zur Anwendung, weil die Bundesländer 
 eigene Datenschutzgesetze haben.
 
 \section{Was sind nicht öffentliche Stellen}
 \begin{itemize}
  \item natürliche Personen (Freiberufler)
  \item juristische Personen des Privatrechts
  \item eingetragene Idealverein §21 BGB (politische Parteien,
  Haus- und Grundbesitzverein...)
  \item wirtschatfliche Vereine (Taxizentralen, Inkassovereine)
  \item Kapitalgesellschaften (AG, GmBH...)
  \item Gesellschaften und andere Personenvereinigungen des privaten Rechts 
  (BGB-Gesellschaft)
  \item falls Erhebung, Verarbeitung oder Nutzung der Daten ausschließlich für
  persönliche oder familiäre Tätigkeiten erfolgt ... 
 \end{itemize}

 \section{Subsidiaritätsgrundsatz}
 \begin{itemize}
  \item Das BDSG findet nur dann Anwedung, wenn keine anderen Vorschriften vorgehen
  (z.B. SGB, AO, Passgesetz, BKA-G)
 \end{itemize}

 
  \section{Die vier wichtigsten datenschutzrechtlichen Grundsätze sind}
 
 \begin{itemize}
 \item Zweckbindung
 \item Erforderlichkeit
 \item Transperenz
 \item Datensparsamkeit
\end{itemize}

\section{Zweckbindung}

Datenverarbeitende Stelle darf personenbezogene Daten nur für festgelegte eindeutige und
\textbf{rechtmäßige Zwecke} erheben und weiterverarbeiten. §13, 14 jeweilse Absatz 1 
28 - Absatz 2\\

Ausnahme anderer Zweck: §14 Absatz 2

\section{Erforderlichkeit}
Erhebung, Verarbeitung und Nutzung personenbezogener Daten 
ist auf das Maß zu beschränken, das für die Erreichung des 
jeweiligen Zweckes notwendig ist. Gibt es eine andere, 
Daten sparsamere Möglichkeit, einen Zweck zu erreichen, ist 
keine Erforderlichkeit gegeben.\\

(Beispiel: Videoüberwachung nicht grenzenlos)  

\section{Transparenz}
Datenverarbeitung muss von Ausnahmen abgesehen für diejenigen von dem Daten verarbeitet werden \textbf{offensichtlich} sein. Er muss über jedes Stadium
Bescheid wissen.\\
Daraus ergeben sich: Informations- und Benachrichtigungspflichten,
§§ 4 Abs. 3, 19a,\\
organisatorische Transparenz (Verfahrensverzeichnis, Datenschutzbeauftragter).

\section{Datensparsamkeit}

Bei einer Erhebung der personenbezogenen Daten ist stets zu prüfen ob mit einer geringeren Datenerhebung und Verarbeitung auch der Zweck erfüllt
werden kann.

\section{Grundsätze des Datenschutzes}
\begin{enumerate}
 \item Datengeheimnis § 5 : Für Personen, die in der Datenverarbeitung beschäftigt sind 
(Rechenzentrum, Personalabteilung). - Verstoß = Wenn personenbezogene Daten unbefugt erhoben, verarbeitet 
oder genutzt werden.
 \item Datensicherheit §9
 \item Datenvermeidung und Sparsamkeit § 3a 
 \item Verbot mit Erlaubnisvorbehalt §4 Absatz 1
 \item Unabdingbare Rechte des Betroffenen §6 Absatz 1
 \item Datenschutzrechtliche Gebote §14 Absatz 1
\end{enumerate}

\section{Was heißt unbefugt - erhoben, verarbeitet oder genutzt?}

Unbefugt handelt, wer weder aus Gesetz, Verordnung, 
Anordnung, Vertrag oder Einzelanweisung eine Erlaubnis für die 
Verarbeitung oder Nutzung ableiten kann. 
Datengeheimnis verletzt auch der, der Vorkehrungen zur 
Datensicherung umgeht. 

\section{Rechtliche Folgen bei Verletzung des Datengeheimnisses?}

\begin{itemize}
 \item zivil
 \item arbeits- oder 
 \item strafrechtliche
\end{itemize}

\section{Verpflichtung auf das Datengeheimnis?}

Bundesdatenschutzgesetz erwähnt nur Beschäftige bei 
nicht-öffentlichen Stellen, weil die des öffentlichen 
Dienstes bereits aufgrund dienst- oder arbeitsrechtlicher 
Vorschriften zur Verschwiegenheit verpflichtet sind.\\
\\
Verpflichtung zur Verschwiegenheit \\
muss bei Aufnahme der Tätigkeit erfolgen.

\section{Was müssen die Verantwortlichen der 
Datenverarbeitung tun ?}

erforderliche organisatorische und technische Maßnahmen umsetzen - für sicherstellung von

\begin{itemize}
 \item Vertraulichkeit
 \item Integrität
 \item Verfügbarkeit
\end{itemize}


\section{Acht Grundsätze der Datensicherheit}
\begin{enumerate}
 \item Zutrittskontrolle
 \item Zugangskontrolle
 \item Zugriffskontrolle
 \item Weitergabekontrolle
 \item Eingabekontrolle
 \item Auftragskontrolle
 \item Verfügbarkeitskontrolle
 \item Trennungsgebot
\end{enumerate}
Anlage (zu § 9 Satz 1)

\section{Wenn Datenerhebung nicht zu vermeiden ist dann falls möglich:}
\begin{itemize}
 \item anonymisiert §3 Absatz 6
 \item pseudonymisiert §3 Absatz 6a
\end{itemize}

\section{ Unabdingbare Rechte des Betroffenen}
\begin{itemize}
 \item Berichtigung, Löschung, Sperrung, Auskunft können nicht
 durch Rechtsgeschäft außer Kraft gesetzt werden.
\end{itemize}

\section{ Verarbeiten von personenbezogenen Daten im Ausland}

Das BDSG unterscheidet drei Fälle von Datenverarbeitung im Auslandsbezug:
\begin{enumerate}
 \item Ausländische Firma verarbeitet vom Ausland im Inland §1 Abs. 5
 \item Deutsche Firma übermittelt ins Ausland § 4b
 \item Auftragsverarbeitung im Ausland §11
\end{enumerate}

\section{Beispiel zu Ausländische Firma verarbeitet vom Ausland (EU) im Inland }
\begin{itemize}
 \item italienische Firma erhebt Daten von in Deutschland tätigen Mitarbeitern
 \item falls die Firma dies nicht über eine Niederlassung tut gilt italinisches
 Recht
 \item Sitzprinzip Es findetdas Datenschutzrecht des Landes Anwendung in dem die
 verantwortliche Stelle ihren Sitz hat.
\end{itemize}
\section{Beispiel zu Ausländische Firma verarbeitet vom Ausland (außerhalb EU) im Inland }
\begin{itemize}
 \item amerikanische Firma erhebt Daten von Mitarbeitern die in Deutschland tätig sind
 \item dann Territorialprinzip: Das anzuwendende nationale Datenschutzrecht richtet sich 
nach dem Ort der Datenverarbeitung, nicht nach dem Sitz 
der verantwortlichen Stelle - also BDSG gilt
\end{itemize}
\section{Datentransfer ins Ausland §4b}

\begin{itemize}
 \item innerhalb EU §4b Absatz 1: Wenn Übermittelung an inländische Stellen erlaubt, dann auch prinzipiell inerhalb der EU.
 \item außerhalb EU §4b Absatz 2: Verfügt ein Land, in das Daten übermittelt werden, über ein 
'angemessenes Datenschutzniveau', dann ist die  
Datenübermittlung zulässig. § 4 b Abs. 2 S. 2 i. V. m. Abs. 3
\\ Bund darf zur Verteidigung, Krisenbewältigung und humanitären Maßhnahmen trotzdem 
falls erforderlich §4b Abs. 2 Satz 3
\end{itemize}

\section{Angemessenheit des Schutzniveau § 4b Abs.3}
\begin{itemize}
 \item Art der Daten
 \item Zweckbestimmung
 \item Dauer der geplanten Verarbeitung
 \item Herkunftsland
 \item Endbestimmungsland
 \item Rechtsnormen, Standesregeln und 
 \item Sicherheitsmaßnahmen, die für Empfänger gelten 
\end{itemize}

\section{Angemessenheit des Schutzniveaus sicherstellen?}
\begin{itemize}
 \item Standardvertragsklauseln (bietet sich für transnationale Unternehmen an)
\end{itemize}

\section{US-Safe Harbor Principles? -Transfer ins Ausland}

\begin{itemize}
 \item Schuld sind die USA- weil die Datenschutztechnisch richtig mies dastehen (kein angemessenes Schutzniveau)
 \item Safe Habor für freien Datenverkehr zwischen Europa und USA (entworfen von US - Handelsministerium)
\end{itemize}

\section{Auftragsverarbeitung}
\begin{itemize}
 \item praktische Bedeutung: Immer mehr Firmen verlagern die Verarbeitung ihrer 
personenbezogenen Daten in andere Unternehmen oder in 
andere, kostengünstigere Länder (z. B. Prag, Indien). Beispiele( Lettershop, Cloud Computing)
\item Grundsatz der Datenverarbeitung: verantwortliche Stelle bleibt Herr der Daten. Auftragsverarbeitung liegt 
nur dann vor wenn Hauptzweck der Auslagerung in der Datenverarbeitung für einen anderen liegt.
\item Pflichten des Auftraggebers: Prüfung Rechtmäßigkeit der Datenverarbeitung §4 Abs. 2 - Führen eines Datenregisters -
Betroffene benachrichtegen §33 Auskunfsverpflichtung §34
\item Pflichten des Auftragsdatenverarbeiter: Datenverarbeitung im Rahmen der Weisung des Auftragsgeber
\item Wahrung des Diensgeheimnisses §5
\item Gewährleistung der erforderlichen technischen und organisatorischen Schutzmaßnahmen
\end{itemize}


\section{Videoüberwachung }

\begin{itemize}
 \item Wann ist sie öffentlich möglich: Aufgabenerfüllung, Wahrnehmung des Hausrechts, Wahrnehmung berechtigter Interessen 
 für konkret festgelegte Ziele
 \item Beobachtung muss kenntlich gemacht werden.
 \item Daten müssen nach erreichen des Zwecks unverzüglich gelöscht werden. (Datenvermeidung und Datensparsamkeit)
 \item schützwürdiges Interesse des Betroffenen ist zu beachten 
 \item werden Daten person zugeordnet - muss Person informiert werden
 \item Hauptproblem - Abwägung Erforderlichkeit zur Erreichung des Zwecks und der Schutzwürdigkeit
\end{itemize}



\section{ Datenverarbeitung der öffentlichen Stelle und nicht öffentliche Stelle}

\begin{itemize}
 \item Paragraphen: 12-18
 \item Datenerhebung §13 : Setzt Aktives Handeln Voraus (speichern ohne erheben möglich bei aufgedrängter information) -
 zulässig falls Zweckerfüllung einer rechtmäßigen Aufgabe und Erforderlichkeit der Maßnahme (d.h vorsorglich dürfen Daten nicht
 erhoben werden) - Erforderlichkeit = objektiv geeignet und auch angemessen - es gilt der Grundsatz der Direkterhebung §4 Abs 2 BDSG + Betroffener muss unterrichtet
 werden Abs 3.\\ 
 wann Erhebung bei Dritten möglich? 1. Rechtsvorschrift 2.  wenn zu erfüllende Verwaltungsaufgabe ihrer Art nach ... Erhebung bei anderen Person erforderlich macht.
 (Straftat) 3. wenn ein unverhältnismäßiger Erhebungsaufwand beim Betroffenen notwendig wäre und kein überwiegend schutzwürdiges Interesse des Betroffenen besteht, \\
 erhebt man Daten besonderer Art dann nur falls - Rechtsvorschrift / Einwilligung /Schutz lebennotwendiger Interessen

 \item Datenspeicherung, - veränderung und – nutzung § 14 : Unterteilt in zwei Kategorieen für Erhebungszwecke §14 Abs. 1 und für andere Zwecke Abs. 2( Erforderlichkeit 
 + Wahrung der Zweckmäßigkeit muss gegeben sein) ... Zweckänderung findet man im Absatz 2 z.B. Arbeitgeber kann nicht beim Finanzamt fragen
 weil Arbeitnehmer Adresse gewechselt hat und er ihm die Kündigung zuschicken möchte.
 \item Datenbübertragung an öffentliche Stelle § 15 - Neue Stelle muss natürlich Vorraussetzungen von 14 erfüllen
 \item Datenübertragung an nicht öffentliche Stelle § 16 - Neue Stelle muss natürlich Vorraussetzungen von 14 erfüllen
 \item Zwei Arten der Datenübertragung - ohne Ersuchen - Datenübertragende Stelle trägt Verantwortung für Rechtmäßigkeit - mit Ersuchen - 
 Fragesteller trägt Verantwortung //Keine Unterrichtung des Betroffenen nötig, wenn damit zu rechnen 
ist, dass Betroffene auf andere Weise Kenntnis von Übermittlung 
erlangt oder Unterrichtung öffentliche Sicherheit gefährden oder Wohl 
des Bundes oder eines Landes Nachteile bereiten würde// übermittelnde Stelle trägt Verantwortung
 \item Wann liegt Datenübermittlung vor? § 3 Abs. 4 Zi. 3: Übermitteln ist das Bekanntgeben gespeicherter oder 
durch Datenverarbeitung gewonnener personenbezogener 
Daten an einen Dritten in der Weise, dass \\   Datenempfänger muss Daten für Sendungszweck verwenden - der von übermittlender Stelle
mitgeteilt wurde
                                                           
die Daten an den Dritten weitergegeben werden oder \\
der Dritte zur Einsicht oder zum Abruf bereitgehaltene Daten einsieht oder abruft.
 \item Zweckänderung §14 Absatz 2 gesetzliche Regelung / Einwilligung / im Interesse + keine Verweigerung / Überprüfung /
 Daten stammen aus allgemein zugänglichen Quellen mit Publikationserlaubnis / erhebliche Nachteile Gemeinwohl / Straftaten
 / Rechtsbeeinträchtigung anderer Personen / wissenschaftliche Forschung\\
 
 \item §18 Durchführung des Datenschutzes
 
\end{itemize}

\section{ Datenerhebung für eine Geschätzwecke}
\begin{itemize}
 \item findet sich alles in §28 wieder
\end{itemize}


\section{ Datenverarbeitung der nicht-öffentlichen Stellen §27 - §38a}
Drei Arten der Datenverarbeitung
\begin{itemize}
 \item Datenerhebung -verarbeitung und -nutzung für eigene Zwecke: Umgang mit Daten zur Erfüllung eigener Geschäftszwecke liegt vor,
 wenn Datenverarbeitung als Mittel zur Abwicklung von Verträgen oder zur Betreuung von Kunden dient § 28
 \item Datenerhebung -verarbeitung und -nutzung für die Übermittelung: 
 \item Datenerhebung -verarbeitung und -nutzung für die Übermittelung in anoymierter Form
\end{itemize}
Unterschied Datenerhebung für eigene bzw. für fremde Zwecke:
\begin{itemize}
 \item eigene: §28 Umgang mit Daten zur Erfüllung eigener Geschäftszwecke liegt vor,
 wenn Datenverarbeitung als Mittel zur Abwicklung von Verträgen oder zur Betreuung von Kunden dient § 28
 \item fremde: §29 Umgang mit Daten zur Erfüllung fremder 
Geschäftszwecke liegt vor, wenn Datenverarbeitung als 
Selbstzweck dient, z. B. Auskunfteien, Detekteien, 
Adresshandel. 
\end{itemize}

\section{ Was ist ein rechtsgeschäftsähnliches Vertragsverhältnis?   }

Das ist ein vorvertragliches und nachvertragliches Vertrauensverhältnis.

\section{ §28 Absatz 2 - die merkwürdige und eigentlich überflüssige Interessenabwägung}

Wie sieht Interessenabwägung nach § 28 Abs. 1 Nr. 2 aus?\\
\begin{itemize}
 \item Datenverarbeitung ist zur Wahrung berechtigter  Interessen der 
  verantwortlichen Stelle erforderlich und 
  \item es besteht kein Grund zur Annahme, dass  schutzwürdige 
  Interessen der Betroffenen am Ausschluss der Verarbeitung 
  oder Nutzung überwiegen. 
\end{itemize}

Paragraph § 28 ist ein fetter Brocken durchlesen und selber katogerisieren ist angemessen...

\section{Rechte der Betroffenen}

\begin{itemize}
 \item Benachrichtigung, Auskunft, Berichtigung, Löschung, Sperrung, Unterlassung, Folgenbeseitigung, Schadenersatz
 \end{itemize}
 
\section{Beauftragter für den Datenschutz}
\begin{itemize}
 \item öffentlicher Bereich: Betrieblicher Datenschutzbeauftragter, Landesdatenschutzbeauftragte, Bundesdatenschutzbeauftragter  
 \item nicht öffentlicher Bereich: Betrieblicher Datenschutzbeauftragter, Staatliche Aufsichtsbehörde des Landes, Bundesdatenschutzbeauftragter
 \item Betrieblicher Datenschutzbeauftragter nötig: zehn Arbeitnehmer deren personbezogene automatisch erhoben werden, oder 20 deren Daten manuell erhoben werden\\
 Es sei denn öffentlicher Bereich da immer nötig
 \item Aufgaben des Datenschutzbeauftragen: Überwachung und Unterstützung - Verantwortung für Zulässigkeit und Sicherheit der 
Datenverarbeitung bleibt bei Unternehmensleitung.
 \item DSB hat Anrecht auf folgende Hilfsmittel: Hilfspersonal, Räume, Geräte, Mittel für Tätigkeit und Schulung, Bratwurst
 \item DSB muss ein Verfahrenverzeichnis zur Verfügung gestellt werden: Das Verfahrensverzeichnis ist eine Sammlung aller 
Verfahrensbeschreibungen der bei einem Unternehmen eingesetzten 
automatisierten Verfahren.  Damit soll die Datenverarbeitung 
transparent gestaltet werden. Der DSB kann auf diese Weise die 
betriebliche Datenverarbeitung kontrollieren.
\item Inhalt Verfahrenverzeichnis im einzelnen: Name oder Firma der verantwortlichen Stelle, Inhaber, Vorstände, Geschäftsführer (...), Anschrift der verantwortlichen Stelle,
Zweck der Datenerhebung, -verarbeitung oder nutzung (Geschäftszwecke, die Daten erforderlich machen), Beschreibung der betroffenen Personengruppe und der 
            diesbezüglichen Daten (Adress-, Gesundheits-,  
            Bonitätsdaten) oder Datenkategorien, Fristen für Löschung der Daten
\item Rechte und Pflichen des Datenschutzbeauftragten: Direkt Unternehmensleitung unterstellt/ Verantwortung für Zulässigkeit und Sicherheit der          
    Datenverarbeitung bleibt bei Unternehmensleitung/ Benachteiligungsverbot / Verschwiegenheitspflicht (Berufsgeheimnis)
    \item Vorraussetzungen Datenschutzbeauftragte: Fachkenntnisse: juristische Kenntnisse, bes. BDSG/  technischer Sachverstand/ insbes. Informatik/ betriebswirtschaftliches Wissen,
	/ soziale Kompetenz zur Gewinnung der Mitarbeiter für Ziele des Datenschutzes.\\
	Zuverlässigkeit: charakterliche Eigenschaften (Verantwortungsbewusstsein,          
           Gewissenhaftigkeit, Belastbarkeit,  Durchsetzungsvermögen) / Kommunikationsfähigkeit / Freiheit von Interessenkonflikten.\\
           mit anderen Worten es gibt in der gesamten BRD gerade mal 10 solcher Leute die alle Vorraussetzungen erfüllen - aber die haben mit sicher kein 
           Interesse daran DSB zu sein
           \item Wie bestellt man einen DSB: Bestellung erfolgt schriftlich - Bestellung erfolgt schriftlich und zwei Formen: interner DSB (Arbeitsvertrag) externer DSB
           per Geschäftsbesorgunsvertrag zu bestellen
           \item DSB kann abberufen werden falls wichtiger Grund vorliegt oder Aufsichtbehörde wegen mangelnder Eignung dies verlangt
           \item DSB - besitzt Kündigungsschutz der nur erlischt falls Gründe für die Kündigung als DSB vorliegen. (Befristestes Arbeitsverhältnis ist ungültig)
           \item DSB ist Mädchen für alles - alle Mitarbeiter dürfen sich an DSB wenden.
           \item Verhältnis DSB und Betriebsrat: BR hat bei Einstellung des DSB Mitbestimmungsrecht, wenn Betrieb mehr als 20 wahlberechtigte AN hat, § 99 BetrVG\\
            wird ein bereits Beschäftigter DSB, dann Versetzung, die 
   mitbestimmungspflichtig ist,  §§ 95 Abs. 3, 99 BetrVG
	
\end{itemize}
\section{Aufsichtsbehörde}
\begin{itemize}
 \item Was kann die anstellen?: Auskunft verlangen (Auskunfverweigerungsrecht des Betroffenen bei Selbstbelastung) Prüfungen und Besichtigungen während der Betriebs- und 
    Geschäftszeit vor Ort unternehmen / Anordnung zur Beseitigung technischer oder organisatorische
    Mängel /  Verbot einzelner Verfahren / Abberufung des Datenschutzbeauftragten   
    \item Wer ist die staatliche Aufsichtsbehörde?: Werden von den jeweiligen Bundesländern bestellt. / Können unterschiedliche Institutionen sein. / Variiert  nach Bundesland (Innenministerium, 
   Regierungspräsidium). /  Aufgabe: Melderegister über Datenverarbeitung bestimmter Stellen führen § 38 Abs. 2 (Dies dient der Öffentlichkeit der Verarbeitung personenbezogener Daten, 
wenn kein Datenschutzbeauftragter vorhanden ist.) 
    
 \end{itemize}
 


\end{document}