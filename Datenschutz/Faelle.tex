\documentclass[a4paper,10pt]{scrartcl}
\usepackage[utf8]{inputenc}
\usepackage{graphicx}
\usepackage{subfig}
\usepackage{amsmath}
\usepackage{geometry}
\geometry{a4paper,left=40mm,right=30mm, top=1cm, bottom=2cm} 
%opening
\title{TGI-1}
\author{}

\begin{document}
\section{Fall 1}

\subsection{Kurzbeschreibung}

Streamen von Film auf kino.to - 500 Euro Strafe + 500 Euro Rechtsanwaltgebühr - exakte Angabe von Tag Uhrzeit 

\begin{itemize}
 \item Wie kommt der Anwalt an die IP-Adresse?\\
 IP-Adresse erwirbt Anwalt über ein am Filesharing beteiligtes Unternehmen - böser Lockvogel.
 \item Woher weiß der Anwalt wer die IP-Adersse verwendet hat?\\
 Auskunftsanspruch im Falle des Urherberrecht §101 Abs. 1, 6 UrhG.
 \item Ist die IP-Adresse datenschutzrechtlich geschützt?\\
 Darüber streiten sich die Datenschutzrechler- falls sie auf
 eine Person beziehbar ist, dann ist sie ein personenbezogenes Datum
 und unterliegt dem Datenschutzrecht.
\end{itemize}

\section{Fall 2}

A kopiert ohne Erlaubnis durch das statistische Bundesamt aus dem Jahresbericht Daten und stellt
sie auf seine Homepage.

Zulässig?\\

Ja weil es sich um veröffentlichte, statistische Daten handelt.

\section{Fall 3}

Student A schreibt von seinem Nachbarn ohne dessen Wissen die Telefonnummer
auf und gibt sie an einem Freund.\\

Zulässig?
Ja - weil Freund sieh § 1 Abs. 2 Nr. 3 BDSG

\section{Fall 4}

Die Polizei in Italien macht von dem Deutschen A gegen dessen Willen einen
Alkoholtest. \\
Zulässig nach BDSG ?
Nein denn italienisches Recht.

\section{Fall 5}
B fährt mit dem Zug und hört, wie A mit einem Kollegem telefoniert. A erzählt,
dass in seiner Firma Personen entlassen werden sollen. Er nennt auch Namen.
B hört genau zu, weil er die Firma kennt. Als A den Zug verlassen hat,
protokoliert B das Telefonat sofort auf dem Laptop. Er will später überprüfen,
ob einer seiner Bekannten, der bei der Firma arbeitet, möglicherlerweise von den
Entlassungen betroffen ist. \\

Was ist datenschutzrechtlich geschehen? \\

A darf nicht laut reden \\

- A verstößt gegen Paragraph 5 BDSG - er verletzt das Datengeheimnis\\

Erheben , Verarbeiten, Nutzen

\section{Fall 6}


B darf die Sachen nicht notieren\\
- laut Paragraph 4 BDSG war das unzulässig - Erhebung Verarbeitung und Nutzung
von Daten sind untersagt. - Weiterverarbeitung natürlich auch
§4 Zulässigkeit der Datenerhebung, -verarbeitung und -nutzung Sie haben nach Ihrem Studium der Informatik einen job in der Firma B bekommen. Die 
Firma hat einen Schwerpunkt Ihrer Tätigkeit in der Datenverarbeitung. Während Ihrer 
Tätigkeit erfahren Sie, dass ein Vorgesetzter von Ihnen heimlich Daten kopiert und 
für private Zwecke verwendet. Er kann dies tun, weil die Sicherheitsmaßnahmen in 
der Firma lax gehandhabt werden. Sie melden den Vorgang einem anderen 
Vorgesetzten, der aber ohne Interesse ist. Nun überlegen Sie, ob Sie die Polizei 
einschalten sollen. Zugleich weisen Sie Ihren Vorgesetzten darauf hin, dass das 
Sicherheitskonzept der Firma mangelhaft ist. Ihr Vorgesetzter kennt sich auch mit 
Sicherheitsfragen nur begrenzt aus und bittet Sie um Rat. \\
\\
1. Dürfen Sie der Polizei von den betrieblichen Unregelmäßigkeiten berichten?\\
2. Was raten Sie Ihrem Vorgesetzten in Sachen Sicherheit der Datenverarbeitung?\\


\section{Fall 7}

A ruft von einem Hotel aus seine Lebensgefährtin an. Später erfährt 
er, dass die Telefonnummer der Lebensgefährtin von dem Hotel 
zum Zwecke der Abrechnung gespeichert wurde.\\
Er hält dies nach § 4 Abs. 1 BDSG für unzulässig 
(mangelnde Einwilligung).\\

Stimmt das ?
Ja, weil weder nach § 4 Abs. 1 BDSG noch gesetzliche 
Grundlage.

\section{Fall 8}

A, der bei einem Kfz-Händler einen Vertrag über den 
Erwerb eines PKW unterschrieben hat, bejaht die vom 
Verkäufer gestellte Frage, ob er mit der Übermittlung 
seiner Daten an einen Adresshändler einverstanden sei.\\ 

Ist die Einwilligung rechtsverbindlich ?\\

Nein nach §4a Absatz 1 muss die Einwilligung schriftlich erfolgen.

\section{Fall 9}

A unterzeichnet anlässlich des Abschlusses eines
Mietvertrags mit einer Wohnungsbaugesellschaft eine 
schriftliche Erklärung. Danach ist er einverstanden, dass 
alle personenbezogenen Daten über ihn gespeichert und 
an nicht näher bestimmte Dritte weitergegeben werden.\\

Ist die Einwilligung wirksam ?\\

Nein da nach §4a der Zweck der Erhebung angeben werden muss. (Erklärung ist zu unbestimmt)

\section{Fall 10}

A unterzeichnet bei seinem neuen Arbeitgeber einen
Formulararbeitsvertrag. Darin erklärt er u. a. sein 
Einverständnis mit der Speicherung bestimmter, auf seine 
Person bezogenen Daten.\\
Ist die Einwilligung wirksam?\\

Nein,\\ 
weil die Einwilligung nicht besonders hervorgehoben ist, \\

§ 4a Abs. 1 BDSG.

\section{Fall 11}

A ist bei einem Unternehmen an einem “sicherheitsempfindlichen” 
Arbeitsplatz beschäftigt. Er erfährt, dass Informationen über ihn im 
Rahmen des sog. Geheimschutzverfahrens an das Bundesamt für 
Verfassungsschutz weitergegeben wurden. \\

Zweck der Weitergabe ist die Prüfung und Entscheidung darüber, ob
ihm von der Behörde die Ermächtigung zum Umgang mit 
Verschlusssachen erteilt wird.  \\

Ist die Weitergabe zulässig?\\

Das hängt davon ab, ob das                             
BundesverfassungsschutzG dafür eine Regelung vorsieht, vgl. §§ 8 ff. ?\\

Wenn nicht, dann müsste eine Einwilligung von A vorliegen.\\

\section{Fall 12}

Ehefrau A hat gegen ihren geschiedenen Ehemann B einen 
Rechtsstreit wegen der Höhe des zu zahlenden Unterhalts anhängig 
gemacht. Trotz Mahnung der ehemaligen Frau weigert sich B, die 
Höhe des zuletzt bezogenen Arbeitslosengeldes mitzuteilen. \\

Daraufhin wendet sich die Frau an das Arbeitsamt, das sie informiert. \\

Darf es das ? \\ da es eine gesetzliche Regelung gibt §§ 35 SGB


\section{Fall 13}
Verarbeitung von personenbezogenen Daten im Ausland\\
Sie haben nach dem Bachelor - Abschluss ein Unternehmen für 
Informationsverarbeitung gegründet und müssen 
kostensparend arbeiten. Deshalb überlegen Sie sich, Ihre 
Datenspeicherung nicht in Ihrer Firma zu machen, sondern 
auszulagern. Sie haben viel von Cloud Computing gelesen und 
möchten dessen finanzielle Vorteile nutzen. Zugleich befürchten 
Sie, dass Ihre Daten nicht sicher verarbeitet werden. Sie prüfen 
deshalb die verschiedenen Möglichkeiten, die Ihnen das BDSG 
dafür bietet. Wie würden Sie einen Cloud Computing-Vertrag 
gestalten? Welches Modell hat welche Vor- und Nachteile?\\ 
Ziehen Sie die §§ 4b und 11 BDSG zu Rate. \\


Kriterien:

4b unter der Lupe:\\

Zwei Fälle wesentlich: 
\begin{itemize}
 \item andere Mitgliedstaaten der EU 
 \item über den europäischen Wirtschaftsraum hinaus
\end{itemize}
Verantwortung liegt bei übermittelnder Stelle - wird zwar nirgends explizit
gesagt - folgt wohl aus der Tatsache das die verantwortliche Stelle
nur für die sichere Übertragung Verantwortung trägt\\
3 Vorteile und 1 Nachteil\\
Im Prinzip hohe Sicherheit über den Vertrag - aber keine Kontrollmöglichkeit\\
niedrige Kosten\\
keine Haftung (bei AG)\\

Sicherheit \\
Kosten \\
Verantwortung\\

11 (heißt im Prinzip der anderen Firma genau vorschreiben wie es zu machen ist
und dann ständig zu kontrollieren - weil man eigene Verantwortung trägt - gilt nur 
innerhalb der EU)
\begin{itemize}
 \item vorteile : starke Einflussnahme, hohe Datensicherheit
 \item nachteile: eigene Haftung, hohe Kosten
\end{itemize}

Lösung 1: 4b + vertragliche Erweiterung für Einflussnahme
Lösung 2: 4b + befristeter Vertrag - mal schauen wie das läuft - Paragraph 11 hat den 
Nachteil das das ganze Verträgeaufsetzen und die ständige Kontrolle erheblicher
Mehraufwand ist.

\section{Fall 14}

Wann ist Videoüberwachung möglich ?\\
Weil in den Hörsälen immer wieder Computer und Beamer 
gestohlen werden, beschließt der Präsident der FH Frankfurt, dass 
in Zukunft die Hörsäle videoüberwacht werden. In jedem Hörsaal 
wird eine Videokamera so angebracht, dass Tag und Nacht der 
Ausgang kontrolliert werden kann. Die Videoaufnahmen werden für 
eine Zeit von drei Monaten von der Liegenschaftsverwaltung der 
Hochschule aufbewahrt. Vorsichtshalber werden sie auch an die 
zuständige Polizeidienststelle übermittelt, damit diese eine Kontrolle 
über das Diebstahlsverhalten an der FH hat.\\

Was sagen Sie zu dieser Regelung? \\

Ist verboten §6b - die Erhebung der Daten für die Erfüllung der Aufgabe
ist nicht erforderlich - da datensparsmare Methode (Alarmanlage oder 
zumindest anbringen innerhalb vom Hörsaal möglich ist).


\section{Fall 15}

Der Eigentümer eines Wohnhauses ist ein vorsichtiger Mensch. Er hat im Parterre 
seines Hauses an an eine Drogerie vermietet, die im Hinterhof den Parkplatz 
benutzen darf. Der Eigentümer fürchtet, dass Kunden der Drogerie über den 
Parkplatz in das Haus gelangen und in seine Wohnung einbrechen können. 
Deshalb installiert er an dem Parkplatz eine Videokamera, die ihm den ständigen 
Blick auf den Parkplatz ermöglicht. Weiterhin installiert er eine Kamera auf dem 
Dach, da er befürchtet, über eine angrenzende Mauer könnten Einbrecher in sein
Haus gelangen. Schließlich bringt er eine Kamera vor der Eingangstür seines 
Hauses an, damit er kontrollieren kann, wer das Haus betritt. \\

Was meinen Sie dazu datenschutzrechtlich?

\section{Fall 16}

Aktueller Fall\\
Fahrradfahrer A ist mehrfach von PKW-Fahrern die Vorfahrt genommen 
worden. Daraufhin beschließt er, um Beweismaterial zu sichern, eine 
Videokamera am Lenker seines Rads anzubringen und während der 
Fahrt zu filmen. \\

Wieder einmal nimmt ihm ein PKW-Fahrer die Vorfahrt und es kommt zu 
einem Unfall. A legt dem Gericht als Beweismaterial das Video vor, auf 
dem das Kennzeichen des Unfallverursachers zu sehen ist und der 
Tathergang sichtbar wird.\\

War die Videoüberwachung zulässig?


\section{Fall 17}

Die Bibliothek der Hochschule Darmstadt hat von Ihnen 
personenbezogene Daten zu Ihrem Leseverhalten erhoben und  
gespeichert:\\
	1. welche Bücher haben Sie ausgeliehen (Autor, Titel,    
         Erscheinungsort),\\
	2. wie häufig haben Sie eine Verlängerung beantragt,\\
	3. in welcher Buchhandlung kaufen Sie bevorzugt Bücher,\\
	4. welche Literatur lesen Sie neben Fachbüchern.\\

Diese Informationen benutzt die Bibliothek für unterschiedliche 
Zwecke:\\
	1.Verwaltung ihres Bücherbestands,\\
	2. Kontrolle des Lernverhaltens der Studierenden (es wird für   
         jeden Ausleihenden eine Matrix erstellt und anschließende eine 
         Rangliste nach verschiedenen Kriterien):\\
		- 1Zahl der Ausleihen,\\
		- fachspezifische und fachfremde Titel, bes.Titel, die einen Zusammenhang zum     
                  Terrorismus aufweisen,\\
		- 3 Häufigkeit des Überschreitens der Ausleihfrist,\\
		- 4 Säumigkeit bei Mahnungen mit Gelddrohung,\\
		- Vorratsdatenspeicherung für mögliche Zwecke der Strafverfolgung,\\
	      - Daten über die Nutzung pornografischer Internetinhalte beim Surfen.\\
	      
Die Bibliothek ist bereit, über diese Daten Auskunft zu geben an 
folgende Stellen:\\

	- die Dekane aller Fachbereiche der Hochschule,\\
	- das Prüfungsamt der Hochschule,\\
	- den Datenschutzbeauftragten der Hochschule,\\
	- anfragende Polizeidienststellen,\\
	- die Ausländerbehörden,\\
	- Buchhandlungen, die den fleißigsten Leser eines Semesters mit 
        einer Buchprämie belohnen wollen.\\
        
 Erhebung zulässig § 13 ? Erforderlich zur Aufgabenerfüllung?\\
 
 erfüllen + erforderlichkeit
 Bücher ausgeliehen ja für Zweck  1, 3, 4\\
 wie häufgi verlängerung ja gleiche Zwecke\\
 in welcher Buchhandlung kaufen sie bevorzugt bücher nein\\
 welche literatur lesen sie neben Fachbüchern nein\\
 \\
 Speichern genau dasselbe §15\\
 \\
 Auskunft\\
 
 an die Dekane unzulässig\\
 Prüfungsamt unzulässig\\
 Datenschutzbeauftragten unzulässig\\
 alles andere auch nein -\\
 
 Buchhandlung aber wegen Paragraph 16
 
 \section{Fall 19}
 
 Der Vorsitzende des Personalrats einer Bundesbehörde speichert auf seinem 
privatem Computer, den er dem Personalrat zur Verfügung stellt, folgende 
Daten:\\
	  - Namen der Beschäftigten und ihre Planstellennummern\\
       - Funktionen und ihre Bewertung\\
       - Tätigkeitsbereiche\\
       - Besoldungsgruppen\\	
	  - Geburts-, Einstellungs- und  Ernennungsdaten der Beschäftigten \\
\\
Kann der Dienststellenleiter die Löschung verlangen ?\\

Aufgabe ist eine Trickfrage -wie leicht überliest man das Wort privat ... also ja weil privater Rechner
 
 
\section{Fall 19 + x}
Sie wollen eine Wohnung mieten. Sie werden von Ihrem künftigen 
Vermieter nach folgenden Daten gefragt: \\
\begin{itemize}
 \item berufliche Tätigkeit Nein
 \item Einkommen Ja
 \item familiäre Verhältnisse Nein, nur Anzahl der Personen
 \item Freundschaftskreis Nein
 \item Hobbies, insbesondere Musik Ja, wenn belästigend laut
 \item Tierhaltung Ja, aber bedingt
\end{itemize}

Sie sehen darin einen Verstoß gegen das BDSG. Wie begründen Sie
dies? Ziehen Sie § 28 BDSG heran. \\

\end{document}