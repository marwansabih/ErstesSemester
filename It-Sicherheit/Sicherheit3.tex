\documentclass[a4paper,10pt]{scrartcl}
\usepackage[utf8]{inputenc}
\usepackage{graphicx}
\usepackage{subfig}
\usepackage{amsmath}
\usepackage{geometry}
\geometry{a4paper,left=40mm,right=30mm, top=1cm, bottom=2cm} 
%opening
\title{TGI-1}
\author{}

\begin{document}
\section{Malware}

\subsection{Malware}

\begin{itemize}
 \item \textbf{Malware:} Für Malicious (bösartig) Hardware, Software oder Kombination von beiden
\end{itemize}

\subsection{Definition}


\begin{itemize}
 \item \textbf{Prozess:} Programm was ausgeführt wird.
 \item \textbf{Betriebssystem:} Erstes Programm was beim Starten ausgeführt wird und dann immer aktiv bleibt.
\end{itemize}

\subsection{Arten von Malware}

\subsubsection{Klassen von Malware (Joanna Rutkowska)}
\begin{itemize}
 \item \textbf{Klasse 0: } Die Malware ändert keinen kritischen Teils des System.
 \item \textbf{Klasse 1: } Malware ändert einen kritischen Bereich, der sich so gut wie nie geändert werden sollte 
 \item \textbf{Klasse 2: } Die Malware ändert einen kritischen Bereich der städig geändert werden darf (unwichtig!) 
 \item \textbf{Klasse 3: } Überhaupt nicht im System
 \item \textbf{Zu Klasse 0: } Böser Prozess der neben anderen Prozessen exitiert (häufigste Form)
                              \begin{itemize}
                               \item Trojaner: sind neue Anwendungen die vom Benutzer ausgeführt werden meistens ohne Wissen, dass 
                               es sich um Malware handelt
                               \item Viren: modifizieren existierende Anwendungen 
                               \item Giftige Eingabe: Nutzen Bugs aus (Exploits) ... wobei nicht kritische Teile des System geändert
                               werden
                              \end{itemize}

 \item \textbf{Zu Klasse 1: } Entweder Änderung im Kern des Betriebsystem - oder Änderungen in wichtigen SystempBetrozessen.
                               \begin{itemize}
                               \item Rootkits: Einbringen von Malware in den Betriebssystemkernel mit dem Ziel die Malware vollständig zu verstecken
                               \item Insider Angriffe (Jemand von innerhalb des System): (Achtung ist nicht immer Insiderangriff) Backdoor(Hintertür ) manipulierte Software die bestimmmten Leuten Zugriff 
                               aus das Betriebssystem ermöglicht\\
                                   Logikbombe wird ausgelöst wen bestimmte Bedingungen erfüllt werden (z.B. bestimmtes Datum) 
                              \end{itemize}
                              
 \item \textbf{Zu Klasse 3: } Z.B. Virtuells System oder Videokamera
\end{itemize}

\subsubsection{Aufspüren von Malware}
\begin{itemize}
 \item Analyse Software
 \item Analyse Hardware (schwer)
 \item Analyse Netzwerkverkehr
\end{itemize}

\end{document}