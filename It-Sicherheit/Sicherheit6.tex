\documentclass[a4paper,10pt]{scrartcl}
\usepackage[utf8]{inputenc}
\usepackage{graphicx}
\usepackage{subfig}
\usepackage{amsmath}
\usepackage{geometry}
\geometry{a4paper,left=40mm,right=30mm, top=1cm, bottom=2cm} 
%opening
\title{TGI-1}
\author{}

\begin{document}
\section{IT-Forensik}
\subsection{Definitionen}

\begin{itemize}
 \item Wissenschaftliche Beantwortung/ Klärung von Rechtsfragen im Kontext von IT-Systemen. 
 \item Es gibt ein Paradigma der Integrität von Beweismittln.
 \item Chain of Custody: Beweiskette einhalten - klare Dokumentation wer wie wann auf das Beweisstück Zugriff hat.
\end{itemize}

\subsection{Locards exchange principles}
\begin{itemize}
 \item Bei Kontakt zwischen Objekten findet immer ein Austausch statt. (Jeder und alles am Tatort hinterläßt etwas und nimmt etwas mit (physische Spuren).
\end{itemize}

\subsection{Anforderungen an IT-Forensik}

\begin{itemize}
 \item Akzeptanz: Nutzung weltweit anerkannten Methoden
 \item Glaubwürdigkeit: Robustheit und Funktionalität der angwandte Methode 
 \item Wiederholbarkeit: liefert immer das gleiche Ergebnis
 \item Integrität: Digitale Spuren bleiben unverändert.
 \item Ursache und Auswirkung: Verbindung zwischen Ereignissen, Spuren und evtl. auch Personen herstelllen.
 \item Dokumentation: Insbesondere Chain of Custody
\end{itemize}

\subsection{Vorgehensmodelle}
S-A-P Modell
\begin{itemize}
 \item Sichern: identifiziern und sichern von Spuren ( Master-, Arbeitskopie)
 \item Analysieren: vorverarbeiten (z.B. nur JPG-Bilder suchen), Inhalte sichten und korrelieren.
 \item Präsentation: dokumentieren für bestimmte Zielgruppe aufarbeiten und vorstellen
\end{itemize}
BSI-Vorgehensmodell:
\begin{itemize}
 \item Strategische Vorbereitung: Vor Eintritt eines Zwischenfall:
 \begin{enumerate}
  \item Bereitstellen von Datenquellen (Logging für Serverdienste etc )
  \item Einrichtung forensicher Workstation samt Zubehör: (Tool, Write Blocker, Kabel für Smartphones ... etc)
  \item Festlegung von Handlungsanweisungen (z.B. Rücksprache mit Jurist)
 \end{enumerate}
 \item Operative Vorbereitung: (Bestandsaufnahme vor Ort nach Eintritt eines Zwischenfalls)
 \begin{enumerate}
  \item Festlegung konkretes Ziel der Ermittelung
  \item Festenlegen der nutzbaren Datenquellen (Datenschutz)
 \end{enumerate}
 \item Datensammlung: (Datenakquise oder Datensicherung):
  \begin{enumerate}
   \item Sicherung der im Rahmen der operativen Vorbereitung festgelegten Daten
   \item Integrität der Daten sowie Vier-Augen-Prinzip berücksichtigen.
   \item Order of Volatilty bei der Datensicherung beachten (z.B Ram zuerst)
  \end{enumerate}
 \item Datenuntersuchen (Vorverarbeitung der anschließnenden Analyse)
 \begin{enumerate}
  \item Datenreduktion (irrelavantes kann weg)
  \iten Datenrekonstruktion
 \end{enumerate}
 \item Datenanalyse (Analyse der vorverarbeiteten Daten)
 \begin{enumerate}
  \item Insbesondere Korrelation der Daten
 \end{enumerate}

\item Dokumetation:
\begin{enumerate}
 \item Verlaufsprotokoll: Einzellschritte auschreiben
 \item Ergebnisprotokoll: Anpassen des Verlaufsprotokoll an Zielgruppe
 \item Nutzung standartersierter Termonologie
\end{enumerate}

\subsection{Datenträgeranalyse}

Erstellen einer Arbeitskopie (Masterkopie,Integrität testen, Anfertigung der Arbeitskopie, Integrität testen)
\begin{itemize}
 \item Paradigma: Orginal so selten wie möglich verwenden (Einfach bei allem Festplattenähnlichen Dingen - schwer bei Smartphones etc.)
 \item Verwendung von Schreibschutz (typischerweise Hardware basierte Write Blocker).
\end{itemize}

\subsubsection{Sektor und Partitionierung}
\begin{itemize}
 \item Sektor: kleinste adressierbare Einheit auf einem Datenträger. Wird adressiert mit LBD (Logical Block Adress) damals (512 Bytes damals) heute (4096 Byte)
 \item Partitionierung: Ziel: Organisation in kleinere Bereiche zur Ablage unterschiedlicher Daten (Linux, Windows) (Partionstabelle)
\end{itemize}
\subsection{Dateisystemanalyse}


\subsubsection{Dateisystem}
\begin{enumerate}
 \item Schnittstelle zwischen Betriebssystem und Datenträger
 \item Verwaltung von Dateien: Namen, Zeitstempel, Speicherort
 \item Phänomen der Fragementierung
 \item Cluster: kleinste adressierbare Einheit auf Dateisystemebene (4096 Byte)
 \item Dateisystemanalyse
 \item Apple benutzt HFS Hierachical File System(sowas wie Fat32 für Windows )
\end{enumerate}


\end{itemize}




 \end{document}