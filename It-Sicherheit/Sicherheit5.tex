\documentclass[a4paper,10pt]{scrartcl}
\usepackage[utf8]{inputenc}
\usepackage{graphicx}
\usepackage{subfig}
\usepackage{amsmath}
\usepackage{geometry}
\geometry{a4paper,left=40mm,right=30mm, top=1cm, bottom=2cm} 
%opening
\title{TGI-1}
\author{}

\begin{document}
\section{Netzwerk und Internetsicherheit}

\subsection{Computernetzwerk}

\begin{itemize}
 \item Zwei oder mehrere Computer, die durch ein
Übertragungsmedium miteinander verbunden (vernetzt) sind, bilden ein Computernetzwerk. 
\item Es gibt Relais Knoten für die Vernetzung da nicht jeder Knoten mit jedem anderen verbunden werden kann (zuviele Verbindungen nötig)
\end{itemize}

\subsection{Adressing und Routing}

\begin{itemize}
 \item Knoten besitzen Adresse
 \item Mit Pfadangabe können dann Pakete über das Netzwerk verschickt werden
 \item Jeder Knoten besitzt Routingtabelle = Tabelle wo hingeschicket werden kann
 \item Forwarding = Prozess der Datenweiterleitung mit Informationen aus dem Routing
 \item Das eigentliche zusammengefasst: Naming = Name für Knoten oder Dienst/ Adressing = Adresse für Knoten oder Dienst / Routing = Bestimmung des Pfades den Daten durch
 das Netzwerk nehmen / Forwarding = weiterleiten von Daten von einem Netzwerksegment ins nächste
\end{itemize}

\subsection{Schichtenmodell}

\begin{itemize}
 \item Es hat sich herausgestellt, dass zur Beherschung der Komplexität von Netzwerken Schichten gleicher Funktionalität nützlich sind.
 \item Schicht ist eine Dienstschnittstelle an die höhere Schicht
 \item Protokoll zum Austausch von Nachrichten
 \item jedes Gerät hat eine MAC Adresse und eine IP-Adresse
 \item beim Austausch werden Pakete übertragen
\end{itemize}

\subsection{IP unter der Lupe}

\begin{itemize}
 \item IP ist Konvergenzprotokoll = überbrückt unterschiedlichste Netzwerktechhnologien / ermöglicht einheitlichen Zugriff für höhere Schichten
\end{itemize}

\subsection{Netzwerksicherheit}

\begin{itemize}
 \item Netzwerksicherheit = Sicherheit aller an das Netzwerk angeschlossenen Geräte abhängig von den Schutzzielen.
 \item Bit und Linklayer = Schichten 1 \& 2: physikalische Verbindung zwischen zwei Netzwerkgeräten - Nur einzelne Segmente können geschützt 
 werden (z.B.) verschlüsseltes WLAN
 \item Network Layer Schicht 3: - Vermittlungsschicht unabhängig von dem gewählten Linklayer - IP-Adresse -Internetprotokoll IPv4 - IPv6 /Ipsec (ist ein VPN) - 
 Alle Geräte inklusive Router sind beteiligt- Probleme mit NATs = Network Adress Translator
 \item Transport-Layer 4: bekanntes Beispiel TLS Transport Layer security = https - beseitig Ipsec Probleme (passiert NATs)- nur direkt zwischen den komunizierenden
 Geräten
 \item Schicht 7: Application Layer - Beispiel: Email - (EzE) Ende zu Ende Sicherheit - bei Email ist EZE über mehrere Hosts verteilt
\end{itemize}

\subsection{Email unter der Lupe}

\begin{itemize}
 \item Drei wichtige Bestandteile: Anwendungsprogramm(Outlook etc.) - Mailserver - SMTP (Simple Mail Transfere Protokoll)
 \item Funktionsweise: Anwendungsprogramm sendet an eigenen Mailserver - dieser speichert in eigene Warteschlange - öffnet TCP Verbindung als Client vom Empfängermailserver -
 Nachricht wird über - Empfängermailserver empfängt Nachricht und speichert sie- Empfänger liest irgendwann die Mail
 \item Übertragung wird in der Regel geschützt - Obwohl die Mails selbst in Klartext auf den Mailservern und Clients liegen.
\end{itemize}3SMTP Client versendet


\subsection{Protokoll-Entwurf}

\begin{itemize}
 \item Wie der Computer Hallo sagt: TCP-Verbindungsanforderung (links) / TCP-Verbindungsbestätigung(rechts)/ GET http://www... (links) / sendet Datei (rechts)
 \item Für Sicherheit: Bei Verbindung feststellen ob Anfrage nicht mit gefälschter IP versendet wurde. (Beispielsweise könnte  Angreifer mit gefälschter IP 1000 Anfragen 
 schicken DOS)
 \item IPsec (Schutzziel Vertraulichkeit): Sender verschlüsselt (IP Payload) - (Schutzziel Authentifizierung:) Zielhost kann Quell-IP authentifizieren
 \item IPsec besteht aus 3 Protokollbestandteilen: 1. Authentication-Header-Protokoll(AH) 2. Encapsulation-Security-Protokoll ESP 3. Internet Key Exchange IKE
 
\end{itemize}

\subsection{Security Association SA}
\begin{itemize}
\item Aufbau einer logischen Verbindung namens 'Security Association' (SA)
\item Verbindung auf der Netzwerkschicht
\item wird für AH und ESP benutzt
\item Eigenschaften: uni-direktional /SA eindeutig bestimmmt durch: Sicherheitsprokoll( AH oder ESP)/ Quell IP/ 32 Bit Verbindungsid
\item 2 Betriebsmodi: Transportmodus (direkte Verbindung zwischen zwei Hosts) / Tunnel-Modus - Tunnel zwischen zwei Hosts - andere Hosts benutzen ihn
\item für Kommunikation von A und B wird eine SA von A nach B benötigt und eine SA von B nach A benötigt
\end{itemize}

\subsection{Was hat AH (Authentication Header) zu bieten?}
\begin{itemize}
 \item Quellen-Authentifizierung/ Datenintegrität/ keine Vertraulichkeit - AH-Heaeder liegt zwischen Daten und IP-Header
 \item gewährt Datenintegrität und Quellen-Identifizierung - aber keine Vertraulichkeit
\end{itemize}
\subsection{Was hat ESP (Encapsulating Security Payload) zu bieten?}
\begin{itemize}
 \item bietet Vertraulichkeit/ Hostauthentifizierung / Datenintegrität
 \item Daten und ESP-Trailer sind verschlüsselt
 \item viel blabla welches unwichtig scheint...
 
\end{itemize}
 \subsection{SA - Management }
\begin{itemize}
 \item ESP oder AH werden für Übertragung verwendet -aber wie baut man die Verbindung auf?
 \item Möglichkeiten: 1. manuelle Konfiguration (selbstständige Eingabe der Verschlüssungsparameter 2. Automatische Konfiguration - Beispiel (IKE = Internet
 Key Exchange)
\end{itemize}

 \subsection{IKE }
\begin{itemize}
 \item Gegenseitige Authentifizierung
 \item Aushandeln der Parameter
 \item Aufbau und halten der Verbindung
 \item benutzt Zertifikate oder shared secrets für den Verbinundsaufbau
 \item Zwei Versionen IKEv1 und IKEV2 (2 scheint super wichtig zu sein)
\end{itemize}

 \subsection{Was die IP alles erlaubt }
\begin{itemize}
 \item Jeder Host kann mit jedem anderen Host komunizieren 
 \item jede Art von Dienst möglich 
 \item Vor- und Nachteile - Angriffe( Denial of Service, Einbrechen in Systeme, Missbrauch von Daten) 
\end{itemize}

 \subsection{Was machen Firewalls? }
 \begin{itemize}
  \item Filtern und untersuchen Datagramme (Hauptsächlich Paketheader (Inhalt würde tiefergehen DPI = Deep Packet Inspection))
  \item Firewalls werden in den Netzwerk-Datenpfad gesetzt
  \item typischerweise an den definierten Netzübergängen (zwischen unterschiedlichen Sicherheitsbereichen)
  \item Firewall erlaubt Durchgang oder blockiert (Einzelne Pakete - bzw. Fluss von Pakten (package flow))
  \item Firewalls blocken sowohl angreifer von außen ab - als auch die Weiterleitung von Viren von Innen.
  \item Firewalls inspezieren die Pakte anhand von Filterregeln (dann aktion pass - bzw. block)
  \item Policiy-Rules implizieren eine Verkehrsfulssrichtung (Ausgehender Verkehr - eingehender Verkehr)
 \end{itemize}

  \subsection{Firewall Typen }
 \begin{itemize}
  \item Nur Ip Layer (seit 2014 nicht mehr im Einsatz)
  \item checkt Layer 4 Tcp und IP header - checkt unter anderem die Semantic ... sehr verbreitet (2014)
  \item Deep Packet Inspection - checkt weiter als Layer 4 bis in die Applikationsdaten hinein
  \item Zustandslose/stateless
  \item Zustandsbehaftete/stateful: behält einen Zustand für eine Verbindung für beispielweise TCP ICMP ...
  \item Unterschied zwischen stateless und stateful - bei stateful gilt wer reinkommt darf auch beantwortet werden
  \end{itemize}

  \subsection{Was ist eine Middlebox? }
  \begin{itemize}

  \item Eine Mittelbox ist alles was zwischen einem Quellhost und einem Zielhost liegt und andere Aufgaben erfüllt als ein normaler Router
  \item Beispiele: Firewall, NAT (Network Adress Translator), Quality of Service Packet markers, Transportverzörgerer und vieles mehr...
  \end{itemize}
  
    \subsection{Middlebox Konfiguration }
   \begin{itemize}
    \item Grundsätzliche Konfiguration: Erlaube ausgehende Verbindungen (statisch) blockiere eingehende Verbindungen (statisch) 
    \item TCP wird von Firewalls bevorzugt da zustandsbehaftetes Protokoll
    \item UDB blockiert oder nur sehr begrenzt zugelassen (Nur für DNS domain name server)
   \end{itemize}

       \subsection{Layout 1: Zwei Arm Mittelbox}
       \begin{itemize}
        \item eine Firwall zwischen dem Intenert und dem Firmennetz
        \item billigste Lösung aber kommt einer rein hat er kompletten Zugriff!
       \end{itemize}
       \subsection{Layout 2: DMZ = Demilitarised Zone }
       \begin{itemize}
        \item Standard Rangehensweise
        \item Externer Webserver FTP Server in DMZ durch Firwall gesichert
        \item Internes Netzwerk nochmal durch eigene Firwall gesichert
       \end{itemize}
       \subsection{Sonstige Layouts}
              \begin{itemize}
               \item Nur eine Middlebox für inneres Netzwerk (fast so sicher wie standardlösung)
               \item Schlechte Lösung zwei interne netzwerke mit jeweils eigener Firwall
               \item bessere Lösung zwei interner Netzwerke die durch selbe Middlebox geschützt werden
              \end{itemize}



 \end{document}