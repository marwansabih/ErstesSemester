\documentclass[a4paper,10pt]{scrartcl}
\usepackage[utf8]{inputenc}
\usepackage{graphicx}
\usepackage{subfig}
\usepackage{amsmath}
\usepackage{geometry}
\geometry{a4paper,left=40mm,right=30mm, top=1cm, bottom=2cm} 
%opening
\title{TGI-1}
\author{}

\begin{document}
\section{Kryptographie}

\subsection{Kommunikationsmodell}

\begin{itemize}
 \item \textbf{Passiver Angriff:} Abhören der Daten 
 \item \textbf{Aktiver Angriff:} Manipulation der Daten
\end{itemize}

\subsection{Schutzziele}

\begin{itemize}
 \item \textbf{Vertraulichkeit:} Verbindung zwischen A(lice) und B(ob) ist sciher O(scar) kann nicht mithören. 
 \item \textbf{Integrität:} Naricht wird nicht verändert, bzw. A und B stellen Änderung fest.
 \item \textbf{Datenauthenzität: } B weiß sicher das Nachricht von A stammt. (Mehrere Emails mit selber IP und unbekannter Autor - 
 fallen in diese Kategorie)
 \item \textbf{Instanzauthenzität: } B kann die Identiät von A zweifelsfrei feststellen.
 \item \textbf{Nichtabstreitbarkeit: } B kann Naricht von A zweifelsfrei auch gegenüber einer dritten Partei als Narich von A 
 nachweisen.
 
 
\end{itemize}
 
 \subsection{Kryptographie für Schutzziele}
 
 \begin{itemize}
 \item \textbf{Vertraulichkeit:} Verschlüsselung 
 \item \textbf{Integrität:} Hashfunktion, Message Authentification Code MAC, Signaturen ( da geb ich dir Brief und Siegel und für)
 \item \textbf{Datenauthenzität: } MAC, Signaturen
 \item \textbf{Nichtabstreitbarkeit: } Signaturen
 \item \textbf{Instanzauthenzität: } Challange Reponse Protokoll ( Hashwert gemeinsam bekanntes Passwort ist der Challange , siehe Beispiel 
 Captcha)
 
\end{itemize}


 \subsection{Auguste Kerckhoff 1883}
 
\begin{itemize}
 \item Ist ein System nicht beweisbar sicher, so sollte es parktisch sicher sein.
 \item Das Design eines Systems sollte keine Geheimhaltung erfordern (Feind darf es wissen).
 \item Ein Kryptosystem muss einfach bedienbar sein.
 \item Ein Angreifer kennt das kryptographische Verfahren, nur die privaten oder symmetrischen Bratwürste sind geheim.
\end{itemize}

\subsection{Symmetrische und Asymmetrische Verfahren}


\begin{itemize}
 \item \textbf{symmetrische Verfahren:} A und B selben Schlüssel, deswegen darf ihn O nicht kennen.
 \item \textbf{asymmetrische Verfahren:} A und B besitzten jeweilse einen öffentlichen sowie einen privaten Schlüssel.
 Die öffentlichen Schlüssel müssen sicher ausgetauscht werden.
\end{itemize}

\subsection{Symmetrische Verschlüsselung im Detail}

\subsubsection{Ceasar}
\begin{itemize}
\item Verschiebt Buchtaben um festen Betrag.
\item Da es nur 26 Mögliche Schlüssel gibt -> Für die Sicherheit muss der Schlüsselraum (Menger möglicher Schlüssel) groß genug sein.
\item Modifiziertes Ceasarverfahren hat 26! Möglichkeiten (SEHR SEHR VIEL) läßt sich aber über relative Häufigkeit knacken.
\item Relative Häufigkeit = Beispiel für statische Angriffe  -> alle Kryptoanalytischen Mehtoden müssen berücksichtigt werden 
großer Schlüsselraum alleine reicht nicht aus.
\end{itemize}

\subsection{One-time Pad}
\begin{itemize}
\item Basiert auf Xor und Bitzahlen (siehe Folige 36-40)
\item Absolut sicher bei Unkenntnis des Schlüssel - jeder sinnvolle Text gleich warhscheinlich.
\item Ciphertext liefert keine Information über den Klartext oder Schlüssel.
\end{itemize}
\subsection{Absolute Sicherheit}
\begin{itemize}
\item \textbf{Absolute Sicherheit: } Wenn ein Angreifer mit unbeschränkten Ressourcen (Zeit, Rechenleistung) das Verschlüsselungsverfahren nicht knacken kann 
- bsp. One-Time-Pad.
\item \textbf{Nachteil} Schlüssel muss genauso lang sein wie Botschaft selbst (Krasses Beispiel 10 GB benötigen 10 GB) Claude Shannon, 1948
\end{itemize}
\subsection{Praktische Sicherheit}
\begin{itemize}
\item \textbf{Praktische Sicherheit: } Schutz vor Angreifer mit beschränkten Ressource
\end{itemize}
\subsection{Modifiziertes One-time Pad}
\begin{itemize}
\item Zweimaliges Anwenden desselben Schlüssel auf unterschiedliche Texte ist nicht besonders effizient - über die Summe der beiden Geheimbotschaften erhält Angreifer die
Summe der Klartexte
\item Man kennt also Summe der Klartexte ohne Schlüssel kennen zu müssen
\item Also müssen wir nur noch zwei sinnvolle Klartexte finden die entsprechende Summe liefern -> selbst kleine Veränderungen des kryptographischen Verfahren können zu einer
massiven Senkung der Sicherheit führen
\end{itemize}

\subsection{Definition Sicherheitsniveau}

\begin{itemize}
\item \textbf{Sicherheisniveau: } Ein Kryptoverfahren hat ein Sicherheitsniveau von n-Bit wenn der Angreifer 2 hoch n Versuche benötigt um es zu brechen.
\item praktische Sicheheit bei größer gleich 100 Bit (heutzutage)
\end{itemize}


\subsection{Stromchiffre}

\begin{itemize}
 \item Aus Schlüssel der größer gleich 100 bit ist wird ein peseudozufälliger Schlüssel generiert
 \item Der Schlüsselstrom wird dann mit dem Klartext addiert (XOR)
 \item erste Idee gleichen Schlüssel mehrmals hintereinander in den Pseudozufallsgenerator - das ist aber unsicher
 \item Aus 100 bit Zufalls schüssel läßt sich mit einem Pseudozufallszahlengenerator natürlich kein echter 200 bit Zufallschlüssel erzeugen -
 Schwierigkeit liegt darin den 100 Bit Schlüssel zu erraten und dann den Schlüsselstrom zu berechnen.
 \item wichtig aus teilen des Schlüsselstroms dürfen keine Vorgänger und Nachfolger bestimmbar sein
 
\end{itemize}

 
 \subsubsection{Stromchiffre:LFSR = Linear Feedback Shit Register}
 
 \begin{itemize}
  \item 1. Schritt Register wird mit zufälligem Schlüssel initialisiert 
  \item Schlüssel wird von links nach rechts durchgeschoben: Was rechts rausfällt ist die Pseudozufallszahl
  \item ausgewählte Bits im Register werden mit XOR addiert und dann rechts eingefügt.
  \item bei einem 11 Bit register lassen sich nach 11 rausgeschobenen Zeichen bei Kenntnis des LFSR ... Nachfolger bestimmen
 \end{itemize}
 
  \subsubsection{Die SUPER POWER: LFSR = Linear Feedback Shit Register}
 \begin{itemize}
  \item Jetzt werden die wesentlichen Schwächen verhindert!
  \item Shrinking Generator: Zwei LFSR R1 und R2 wenn R1 1 ausgibt wird Ausgabe R2 genommen falls R1 0 ausgibt wird sie verworfen.
  \item Summations Generator: Easy Peasy - R1 und R2 addiere einfache die Ausgaben.
  \item boah jetzt haben wir potentielle Angreifer krass verwirrt Manipulation
  \item Beispiele: Krasses Handy 1990 / Sprachverschlüsselung zwischen Mobiltelefon und Funktmast
  
 
 \end{itemize}

 \subsection{Blockchiffre}
 
  \begin{itemize}
   \item Blockchiffren bilden Bistrings fester Länge auf Bitsrings fester Länger ab
   \item moderner Blockchiffren verarbeiten Bitstrings der Länge 128 (Klartextblock)
   \item Genutzte Schlüssel müssen eine Bitlänge >= 100 haben 
   \item Für längere Klartexte (128 Bit) werden sogenannet Betriebsarten genutzte
   \item Zwei Arten: Feistel Chiffren - Prominenentes Beispiel DES und Substitutions-Permutations-Netzwerk SPN - Prominentes Beispiel AES nachfolger von Design -
   wenden beide im Prinzip dasselbe an: 
   \begin{itemize}
   \item Diffusion: Verteilung der Information des Klartext über den gesamten Chiffretext\\
    jedes Bit des Chiffretext hängt von jedem Bit des Klartext ab\\
    Bei Änderung eines Klartextbits ändern sich 50\% der Geheimtextbits\\
    wird durch Permutationen realisiert
   \item Konfusion: Zusammenhang zwischen Klartext und Chiffretext soll hochgradig komplex sein
   gleiches gilt für Schlüssel und Chiffretext
   -realisiert durch Subsitituionen (nicht lineare Abbildung)
   \item Rundenbasiert: Wiederholte Anwendung von Diffusion, Konfusion und Schlüssel
   \end{itemize}

  \end{itemize}
  
  
   \subsection{Diffusion und Konfusion unter der Lupe}
   
   \begin{itemize}
    \item Reihenfolge der Bits werden über den gesamten Block (128 Bit) vertauscht + mehrere Runden nötig damit Diffusion zum tragen kommt
    \item S-Box = nichtlineare Abbilung\\
    KLARTEXTBLOCK (128 Bit)                             1) Aus Schlüssel der länge 128 bit werden n Rundenschlüssel der länge 128 Bit hergeleitet
    
    dann wird der erste Runden Schlüssel auf den Klartext addiert (XOR) dann wenden wir die nicht lineare S-Boxen(haben die länge 8-16 bit) auf den Schlüssel an und anschließend permutieren wir das ganze,
    dies wiederholen wir n -mal und erhalten den Chiffretext\\
    \textbf{Lassen wir die Substitution weg - ist die Blockchiffre eine lineare Abbildung} - deswegen lineares Gleichungssystem und effizient lösbar\\
    \textbf{Lassen wir Permutionen Weg} hängt Bit 1- 16 des Geheimtextes nur nur von Bit 1-16 vom klartextblock ab damit (Attacke auf 1 Bit 17 usw über monoalphabetische
    Verfahren möglich)
    \end{itemize}
    \subsection{Betriebsarten}
   \begin{itemize}
    \item Blockchiffren: verschlüsseln zunächst nur Klartexte einer bestimmten Blockgröße - heute meist 128 Bitlänge
    \item Für längere Klartexte wird der Text in entsprechende Blöcke, falls der letzte Block zu klein ist wird dieser mit einem Padding aufgefüllt
   \end{itemize}

   \subsubsection{ECB-Mode = Electronic code Book}
   \begin{itemize}
    \item Plaintext + immer gleich Key auf jedem Block mit Blockcipher encryption
    \item gleiche Klartextblöcke werden zu gleichen Chiffretexte verschlüsselt
    \item AES im ECB = Wenn man Pinguin kommt Pinguin etwas schwarz wieder raus
    \item aus den Punkten davor folgt, dass wir weitere Parameter und nicht nur den Schlüssel und Klartext zur Erstellung des Chiffretextes benötigen
    \item daraus folgt wir verwenden andere Betriebsmodi CFB und OFB
   \end{itemize}
   
      \subsubsection{CBC-Mode = Cipher Block Chaining}

   \begin{itemize}
    \item Initialisierungsvektor IV - wird benutzt und vor Verschlüssung auf Plaintext (summiert XOR) angewendet (um Rückschlüsse auf Key und Klartext zu verhindern)
    \item entstandener Ciphertext wird als nächste IV für den Block benutzt
    \item Entschlüsselung entprechen umgekehrt
   \end{itemize}

  
      \subsubsection{Counter Mode CTR}
      
      \begin{itemize}
       \item Zuerst erzeuge Wertepaar Nonce = Zuffalls Number only used once und verknüpfe sie mit Counter für jeden Block (z.B. Counter 0000 für erst nächster Block hätte Counter
       0001 usw.) die Verknüpfung von Nonce mit der Counter erfolgt beispielsweise über simples dranhängen oder über Addition (XOR)
       \item Mit Hilfe der Block-Cipher Encrytion und dem Key wird dann aus diesen Werten ein Schlüsselstrom erzeugt
       \item Addition auf den Plaintext ergibt dann entsprechenden Ciphertext - zur Entschlüssung muss dann natürlich entsprechender Schlüsselstrom einfach
       auf den Cipher addiert werden.
       \item Damit dieser Modus sicher ist muss die Blockchiffre -1.) Die Elemente müssen Permutiert werden -> kleine Änderungen (Zähler) führen zu großer Änderung Ciphertext und
       Blockchainig muss nicht linear sein damit auf den Schlüssel nicht geschlossen werden kann - Zusammenfassung nicht linear für Schlüssel - Permutation für Eingabe in die Chiffre
      \end{itemize}
      
      \subsubsection{Was passiert mit dem Pinguin?}
      \begin{itemize}
       \item ECB und CTR töten Pinguin
      \end{itemize}

      \subsection{Hashfunktionen}
      \begin{itemize}
       \item Ziel einer Hashfunktion Integritätschutz
       \item Bilden Bitstrings beliebiger Länge auf Bitstrings fester Länge ab.
       \item Für Datenintegrität
       \item Hashfunktion ist kryptisch stark wenn: Preimage resitance: Es ist praktisch unmöglich einen Eingabeparameter zu finden der einen entsprechnden Hashwert liefert.
                                                    Collision resistance: Es ist praktisch unmöglich zwei Eingabeparameter so zu finden das sie denselben Hashwert liefern.
       \item H soll effenzient berechenbar sein, die Umkehrfunktion soll aber praktisch nicht berechnbar sein (Einwegfunktion)
       
       \item Untersschied Hash und MAC - Hash für Datenauthenzität - MAC für Datenauthenzität und Integrität
       \item Wie funnktioniert das Ganze Merckle Damgard Konstruktion: man könnte z.B. f als blockchiffre nehmen und dann ein Initialisierungsvektor + den Block in f reingeben - was da rauskommt stecken wir
       mit dem nächsten Klartextblock + den Chiffreblocktext wieder in f rein und wiederholen das Ganze bis wir so den ganzen Text durchgegangen sind - der letze Cipherblock
       ist dan der Hashwert.
       \item Beispiel Download von Datei - Hashwert soll stimmen
       \item da Hashfunktionen öffentlich zugänglich sind - läßt sich der Absender einer Naricht nicht ermittel ( Schutzziel Datenauthenzität ist nicht gewährleistet)
               \end{itemize}
      
       \subsection{Macfunktionen}
       \begin{itemize}
        \item Bei symmtrisscher Verschlüsselung (nur A hat k und B hat k) liefert der mac(k,m) eine Prüfsumme mit der B feststellen kann, dass die Nachricht wirklich von A stammt
        \item so erfüllte es aber Nichtabstreitbarkeit weil nur A und B - den Schlüssel k kennen (dafür werden Signaturen benötigt)
        \item mac wendet blockchiffren an CBC-MAC, XOR-MAC (kryptische starke Hashfunktion = HMAC)
        \item CBC Geht für Mac - aber CTR geht nicht für MAC ( weil der Blockweise auswertet)
       \end{itemize}
       \subsection{Secure Messaging}
        \begin{itemize}
         \item Ziel Sicherer Kanal zwischn Alice und Bob
         \item Sowohl Mac als auch Klartext werden verschlüsselt (IPsec)
         \item MAC wird zuerst über Klartext geschickt und dann folgt der verschlüsselte Klartext (SSH)
         \item MAC wird zuerst über Klartext geschickt und Klartext mit Mac nochmach verschlüsselt nachgeschickt (SSL)
         \item Nur IPsec ist sicher
         \item für Mac und Klartext zum berschlüsseln andere Schlüssel \textbf{(Trenne wo du trennen kannst.)}
        \end{itemize}
       \subsection{Instanzauthentisierung}
       \begin{itemize}
        \item Verfahren zur Intanzauthentisierung sind: Benutzername/Passwort - biometrische Merkmale/Fingerabdruck - Challenge-Response Verfahren
       \end{itemize}
       \subsection{Challange Response Verfahren (symmetrisch)}
       \begin{itemize}
       \item Ziel: Nachweises eines Geheimnisses ohne dies offen zu legen
       \item A und B proudly owning the same key A schickt challange = (Zufallszahl) -B bildet MAC von challange mit key und schickt dies A zu A überprüft ob die MAC wirklich von
       dem key stammt.
       \end{itemize}
       \subsection{ Asymmetrische Verfahren} 
       \begin{itemize}
        \item Es gibt zwei öffentliche Schlüssel $pk$ zwei private Schlüssel $ps$.
        \item Mit sk wird die Prüfsumme berechnet mit - pk kann dann Prüfsumme von jedem überprüft werden (IA= Instanzauthentisierung)
        \item Grundidee asymmeterischer Verfahren: Eingwegfunktion mit Falltür - f ist effizient berechnbar - Umkehrfunktion nur mit Zusatzswissen berechenbar)
        \item pupliziert 1977 Rivest, Shimir und Adlemann - erste bekannte asymmetrische Verfahren / erfinder wurden reich - aber es hat etwas gedauert
        \item siehe mathevorlesung
        \item $c = m^e \% N$, $m = c^d \% N$ 
       \end{itemize}
       \subsection{ Hybridverfahren}
       \begin{itemize}
        \item A verschlüsselt mit öffentlichem Schlüssel von B - einen random Key und schickt diesen B zu - nun kann symmetrisch weiterverschlüsselt werden.
        \item A muss sich sicher sein, dass der Schlüssel zu B gehört.
       \end{itemize}
        \subsection{ Signaturverfahren}
       \begin{itemize}
        \item A signiert m mit privaten Schlüssel und schickt Signatur sowie Nachricht m zu B - B kann aus der Signatur über den öffentlichen Key m zurückgewinnen und mit dem 
        zugeschickten m abgleichen.
        \item Bei Signaturverfahren nach RSA - wird der Hashwert der Naricht verschlüsselt. - über öffentlichen Key kriegt man also Orginal Hash der Nachricht
        \item erfüllt die Nichtabstreitbarkeit nur der Inhaber des privaten Schlüssel kann Nachricht signiert haben.
       \end{itemize}
       \subsection{ challange Response - Asymmetrisch}
       \begin{itemize}
        \item Nachweis eines Geheimnisses ohne dies offen legen zu müssen - schicke challange c an B - mit sk verschlüsselt zurückbekommen - mit pk gewinnt man wieder das Geheimnisses
        - B kennt also sk
       \end{itemize}
        \subsection{ RSA - Zusammenfassung}
        \begin{itemize}
         \item Verschlüsselung
         \item Datenauthentiesierung (Signaturverfahren)
         \item Instanzauthentisierung (Challange-Response-Verfahren)
         \item Für alle Verfahren - werden unterschiedliche Schlüssel eingesetzt (Trenne wo du trennen kannst) sonst sind Angriffe möglich.
         \item Später Beispiel angeben... .
        \end{itemize}
        \subsection{ Vertrauensmodelle }
        \begin{itemize}
         \item Problem 1 + 2: - Nutzung öffentlicher Schlüssel falscher Person - keine Vertraulichkeit mehr gewahrt / falsche Zuordnung des öffentlichen Schlüssel 
         zu einer Person - Datenauthenzität ist nicht gewährleistet
         \item Direct Trust: Nutzer erhält den öffentlichen Schlüssel direkt vom Schlüsselinhaber (kleine Virtual Private Networks)
         \item Web of Trust: Nutzer signieren sich gegenseitig ihre öffentlichen Schlüssel (PGP, GNU-PG) Also Bob signiert Carls öffentlichen Schlüssel - Alice
         vertraut Bob nur vernünftige Leute zu signieren - Alice prüft mit Bobs öffentlichen Schlüssel den öffentlichen Schlüssel von Carl -jedem Schlüssel zugeordnet sind:
         Name des Schlüsselinhabers/ Key Legitimacy (3 Stufen, Grad des Vertrauens in den öffentlichen Schlüssel) leitet sich ab aus den Owner Trusts der Signierer und der Anzahl der Signaturen
         /Signaturen des öffentlichen Schlüssels (Schlüsselbund) - verwentet bei OpenPGP
         \item Owner Trust besitzt 5 Stufen: unbekannt, kein Vertrauen, geringes Vertrauen, volles Vertrauen, absolutes Vertrauen (eigener Schlüssel)
        \end{itemize}
        \subsubsection{ Nachteile Web of Trust }
        \begin{itemize}
         \item Owner Trust - erfordert hohes Wissen der Nutzer
         \item Signaturen sind juristisch nicht bindend
         \item Zurückziehen Zertifikate nicht einfach umsetzbar
        \end{itemize}
        \subsubsection{ Vorteile Web of Trust }
        \begin{itemize}
         \item Deutschliche Verbesserung gegenüber Direct Trust
         \item Umgesetzt in PGP (Pretty good Privacy) und OpenPGP
         \item Zahlreiche Schlüsselserver für Signaturen und öffentliche Schlüssel
         \item Einige Server bieten Certification Service für PGP und GPG Schlüssel an
        \end{itemize}
        
         \subsubsection{ Hierachical Trust }
         \begin{itemize}
          \item Funktioniert über die Stelle die Zertifikate ausgibt CA = Certification Authority
          \item Aufgabenteilung in RA = Registration Authority (Sicherheitsrichtlinien, Zertifikatsanträge, Inhalte der Zertifikate, Leitet Anträge an CA weiter) und CA = Certification Authority
          (stellt eigentlichen Zertifikate aus)
          
          \item Certificate enthält öffentlichen Schlüssel des Nutzers - dieser schickt bei Übertragung
          öffentlichen Schlüssel der CA (der bekannt sein sollte)- sowie message und Signatur
          \item Certificate können entzogen werden Certificate Revokation Lists (CRL)
          \item All Instanzen einer PKI (Root-CA, Teil-CAs, Endnutzer) müssen ein definiertes Maß an
Sicherheitsstandards einhalten.
          \item Certificate Policy (welche Sicherheitsvorgaben müssen eingehalten werden) 
          + Certificate Practise Statement( wie werden die Sicherheitsvorgaben umgesetzt) : beschrieben in RFC 3647
          \item Zertifikate müssen: Inhaber eindeutig identifizieren, Schlüsselnutzung festlegen, Ausstellende CA identifizieren - zwei Standards:
          X509: Eingesetzt zur sicheren Kommunikation im Internet (https) - Card Verifiable Certificates (cvc): Eingesetzt zur sicheren Kommunikation
zwischen/zu Smartcards
           \item OID = Knoten in einem hierachisch zugewiesenem Namensraum - Folge von Nummern die Position beginnend von Wurzel beschreibt(damit wird z.B. der umgesetzte Algorithmus
           beschrieben)
           \item  Zertifikate haben momentan  3 Versionen - aber v3 dann extrainfos (Extenstions bsplweise infos : issuer und subject(Telefon email etc.)
          
         \end{itemize}


                





\end{document}