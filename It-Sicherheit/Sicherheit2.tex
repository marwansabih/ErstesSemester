\documentclass[a4paper,10pt]{scrartcl}
\usepackage[utf8]{inputenc}
\usepackage{graphicx}
\usepackage{subfig}
\usepackage{amsmath}
\usepackage{geometry}
\geometry{a4paper,left=40mm,right=30mm, top=1cm, bottom=2cm} 
%opening
\title{TGI-1}
\author{}

\begin{document}
\section{Privacy by design}

\subsection{Privacy}

\begin{itemize}
 \item \textbf{Privacy:} Privatsphäre = salopp Kontrolle über die eigenen Daten haben
und behalten 
\end{itemize}


\subsection{Ziele von Privacy by design}
\begin{itemize}
 \item \textbf{Privacy by Design: } Gewährleistung eines allumfassendes Datenschutzes  (von Stunde 0 = von Beginn an mit einplanen)
 \item \textbf{proaktiv: } Vorbeugende Maßnahmen 
 \item \textbf{Datenschutz als Standardeinstellung: } Standardmäßige so konfiguriert das Datenschutz erfüllt wird  
 \item \textbf{Datenschutz im Design: } Wirklich das gesamte System betrachten und bei Stunde 0 anfangen.
 \item \textbf{Volle Funktionalität: } Funktionalität soll trotz Privacy gewährleistet sein.
 \item \textbf{Durchgängiger Datenschutz: } Während des gesamten Lebenszyklus gewährleistet ( auch beim Vernichten)
 \item \textbf{Sichtbarkeit und Transparanz} Verfahren sind öffentlich bekannt.
 \item \textbf{ Die Wahrund der Privatsphäre der Nutzer:} Betreiber und Architekten stellen sicher das Interessen von Nutzern gewahrt sind - 
 Datenschutz + benutzerfreundlich 
\end{itemize}


\end{document}